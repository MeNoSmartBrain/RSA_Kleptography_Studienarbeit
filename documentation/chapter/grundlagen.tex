\chapter{Grundlagen}

In diesem Kapitel werden die theoretischen Grundlagen der Kryptologie, Mathematik, Komplexitätstheorie und der IT-Sicherheit erläutert, die in dieser Arbeit eine Rolle spielen. Aus diesen sollen die grundlegenden Funktionen eines kleptographischen Angriffes und dessen Folgen abgeleitet werden.

\section{Kryptologie}
    Die Kryptologie ist die wissenschaftliche Disziplin für den Schutz von Daten. Unter ihr stehen die zwei Felder der Kryptographie und der Kryptoanalyse.

    \subsection{Kryptographie}
        Die Kryptologie befasst sich mit der Entwicklung von Verfahren und Techniken für den sicheren Austausch von Daten. Dabei stehen zwei Eigenschaften im Fokus:

        \subsubsection{Eigenschaften}
            \paragraph{Geheimhaltung}
                Durch Geheimhaltung sollen, bei der Übertragung von Daten zwischen Teilnehmern, Unbeteiligte keine Erkenntnisse über den Inhalt erlangen. Dies kann durch physikalische oder organisatorische Maßnahmen erreicht werden, wobei Unbeteiligten der Zugang zu den übertragenen Daten verwehrt wird. Diese Maßnahmen sind sinnvoll bei der Übergabe der Daten in einer nicht digitalen Welt. Bei der Kommunikation in digitalen Netzen, wie u.a. dem Internet, sind diese Maßnahmen nur schwer zu implementieren. Dies gilt nicht für kryptographische Maßnahmen. Dabei ist es nicht mehr das Ziel, Unbeteiligten den Zugang zu den übertragenen Daten zu erschweren, sondern den Inhalt der Daten während der Übertragung zu verschlüsseln. Dadurch soll es Unbeteiligten nahezu unmöglich sein, aus den mitgehörten oder abgefangenen Daten, Rückschlüsse auf deren Inhalt zu erlangen. \footcite[1]{BSW.2015}
                
            \paragraph{Authentifikation}
                Durch Authentifikation soll es den Teilnehmern einer Kommunikation möglich sein, die anderen Teilnehmer und empfangene Nachrichten zweifelsfrei identifizieren und zuweisen zu können. Hierbei spielen Signaturverfahren eine wichtige Rolle, da kein Geheimnis benötigt wird um einen Teilnehmer zu authentifizieren. Dabei können sich Teilnehmer durch das Wissen oder den Besitz eines Geheimnisses (Passwort, Zertifikat, Schlüssel) authentifizieren. \footcite[2]{BSW.2015}

        Nur wenn beide Eigenschaften gegeben sind ist eine Übertragung von Daten als sicher anzusehen. Falls die Geheimhaltung fehlt, kann der Inhalt durch Sniffing mitgelesen werden. Falls die Authentifikation der Teilnehmer fehlt, können sich Unbeteiligte als "echte" Teilnehmer ausgeben und somit die Daten an ihrem Endpunkt entschlüsseln.

        \subsubsection{Zusätzliche Eigenschaften}
            \paragraph{Perfect Forward Security} Perfect Forward Security

        \subsubsection{Kryptographische Verfahren}
            Kryptographische Verfahren sind Algorithmen, welche die Genheimhaltung von Daten und die Authentifikation von Teilnehmers und Nachrichten sicherstellt. Dadurch kann man sie in Verschlüsselungsverfahren und Authentifikationsverfahren unterscheiden. Dabei können Verfahren, wie z.B. \ac{RSA} beiden Aufgaben übernehmen.
        % Was sind Verschlüsselungsverfahren, Ziel, Historie, momentaner Stand, State of the Art (AES, RSA )
            \paragraph{Asymmetrische Verschlüsselung}
            % Hier angesprochene Verfahren, unterschiede, Definition / Abgrenzung zu symmetrischen Verfahren
            \label{Asymmetrische Verschlüsselnung}
                Bei asymmetrischen Verschlüsselungsverfahren wird statt dem gleichen Schlüssel für das Ver- und Entschlüsseln, zwei verschiedene Schlüssel verwendet. Dabei hat jeder Teilnehmer ein öffentlichen Schlüssel $e$ und einen privaten und geheimen Schlüssel $d$. Hierbei ist es vorgesehen, dass möglichst alle potenziellen Teilnehmer den Schlüssel $e$ kennen. Wenn eine Nachricht mit einem der beiden Schlüssel chiffriert wurde, kann nur mittels dem anderen Schlüssel dechiffriert werden. Somit können Nachrichten an einen Teilnehmer verschlüsselt versandt werden, indem diese mit dem öffentlichen Schlüssel $e$ des Teilnehmers chiffriert wird. Nun kann nur der Teilnehmer mit dem zugehörigen privaten Schlüssel $d$, die Nachricht verschlüsseln. Zusätzlich kann auch die Authentifikation von Teilnehmer und die Authentizität von Nachrichten mit hoher Sicherheit festgestellt werden. Somit kann der Author einer Nachricht, einen Fingerabdruck dieser Nachricht mit seinem privaten Schlüssel $d$ signieren, an die Nachricht anhängen und dann beide Teile verschlüsseln. Diese Signatur kann verifiziert werden, indem der Empfänger die Nachricht entschlüsselt und dann die Signatur verifiziert, indem er den öffentlichen Schlüssel des Authors $e$ auf diesen anwendet. Danach vergleicht er den empfangenen Fingerabdruck mit einem eigens erstellten Fingerabdruck. Somit kann die Geheimhaltung, Authentizität und Integrität der Nachricht bestimmt werden.
                
                Die zwei Schlüssel $e$ und $d$ eines Teilnehmers, werden auch als Schlüsselpaar bezeichnet. Ein solches verfahren, wird asymmetrisch genannt, da für das Ent- und Verschlüsseln zwei unterschiedliche Informationen vorliegen müssen. Diese Informationen sind auch nicht auseinander ableitbar, wie es z.B. bei multiplikativen Chiffren der Fall wäre. Die Funktionalität des Verfahrens, beruht auf der Annahme, dass alle Teilnehmer Zugang zu den öffentlichen Schlüssel jedes anderen Teilnehmers haben bzw. haben können. Durch diese Charakteristika werden solche Verfahren auch als Public-Key-Kryptographie bezeichnet.

                Dabei wird stets die Annahme getroffen, dass der private Schlüssel eines Teilnehmers ausschließlich diesem vorliegt. Anderenfalls ist die Geheimhaltung und die Authentifikation beim Informationsaustausch von und mit diesem Teilnehmer nicht mehr gewährleistet. Somit wäre die Sicherheit kompromittiert.
                
                Die Schlüssel eines Schlüsselpaar bilden somit Umkehrfunktionen zueinander.

            \paragraph{Hybride Verfahren}
            \label{Hybride Verfahren}
                % Meiste Verfahren heute hybride Verfahren. Asymmetrisch zum Schlüsselaustausch. Symmetrisch zum Verschlüsseln. Laufzeit von Asymmetrischen Verfahren. Wenn Schlüsselaustausch sicher Geheimhaltung verletzt.
                Asymmetrische Verschlüsselungsverfahren haben häufig den Nachteil, dass die deutlich rechenaufwändiges sind, wie wir später bei \ac{RSA} sehen werden. Es liegen zwar effiziente Verfahren vor, um z.B. die modularen Potenz aus zwei 300-stelligen Zahlen und einem Modulo zu bilden \ref{sec-Effiziente Berechnung der diskreten Exponentialfunktion}. Dennoch sind diese Verfahren mit mehr Aufwand verbunden, als z.B. symmetrische Blockchiffren wie \ac{AES}.
                
                Deshalb werden asymmetrische Verfahren für die Initialisierung der Kommunikation verwendet. In dieser Initialisierungsphase soll der Teilnehmer authentifiziert werden und ein gemeinsamer, geheimer, symmetrischer Schlüssel vereinbart werden. 
                
                In der darauffolgenden Kommunikationsphase werden die Nachrichten durch symmetrische Chiffren mittels des vereinbarten Schlüssels, effizient verschlüsselt und auf der Gegenseite entschlüsselt. Asymmetrisch Chiffren werden hier benutzt um Fingerabdrücke von Nachrichten zu signieren und zu verifizieren, wie oben \ref{Asymmetrische Verschlüsselnung} gezeigt. Zusätzlich werden durch asymmetrische Verfahren regelmäßig neue symmetrische Schlüssel vereinbart.

                Solche hybriden Verfahren sind z.B. beim Browsen im Internet zu finden. Hier ein Beispiel: \textbf{$TLS_ECDHE_RSA_WITH_AES_128_GCM_SHA256$}


    \subsection{Kryptoanalyse}
        Moderne kryptographische Verfahren, werden nach ihrer Sicherheit und ihrer Effizienz beurteilt. Dabei kann die Sicherheit eines Verfahrens auf mathematische Probleme gestützt werden, welche aktuell und in der nahen Zukunft nicht trivial lösbar sind. 

        Alle Angriffe auf kryptographische Verfahren, gelten dem Erlangen des Geheimtextes einer chiffrierten Nachricht oder dem Berechnen, des verwendeten Geheimnisses.

        Bei Angriffen auf das Geheimnis werden hier lediglich computergestützte Angriffe betrachtet, also nur Angriffe, die durch den Einsatz von Rechnerressourcen und Algorithmen versuchen, den geheimen Schlüssel zu bestimmen. Es wird im Allgemeinen davon ausgegangen, das Nutzergeheimnisse, wie Passwörter, Zertifikate und private Schlüssel nicht öffentlich zugänglich sind.
        Brute-Force Angriffe sind beispielhaft für die Angriffe. Hierbei wird systematisch der Zahlenraum (bzw. Zeichenraum) aller möglicher Schlüsselkombinationen durchprobiert. Weiterentwicklungen dieses Angriffs versuchen auf Grundlage von statistischen Erkenntnissen den Zahlenraum des Geheimnisses einzugrenzen, wie z.B. Wörterbuchangriffe. Hierbei ist die Länge und die Zufälligkeit des Geheimnisses der entscheidende Faktor für einen effektiven Schutz vor Angriffen.
        
        Um die Sicherheit von kryptographischen Verfahren beurteilen zu können, werden Angriffe auf diese Verfahren, nach den hierfür notwendigen Voraussetzungen, unterteil. \footcite[20]{Beutelspacher.2015}
        \paragraph{Known Cipher Attack}
            Hierbei benötigt der Angreifer beliebige Menge an verschlüsselten Text, um aus diesem den Schlüssel und somit den Geheimtext ableiten zu können.
        \paragraph{Known Plaintext Attack}
            Bei diesen Angriffen, kennt der Angreifer eine echte Teilmenge der verschlüsselten Textes und den dazugehörigen Geheimtext. Diese Angriffe sind erfolgreich, wenn sich aus einer echten Teilmenge des Klartextes und dem dazugehörigen Geheimtext, der verwendete Schlüssel berechnen lässt.  
        \paragraph{Chosen Plaintext Attack}
            Bei Chosen Plaintext Attack kann der Angreifer die Chiffre zu einem von ihm gewählten Klartext berechnen. Dies ist ein klassischer Fall bei Public-Key-Kryptographie \ref{Asymmetrische Verschlüsselnung}, da hier der Algorithmus (Prinzip von Kerckhoff) und die Schlüssel öffentlich sind. Somit kann sich eine Angreifer zu jedem beliebigen Klartext die entsprechende Chiffre berechnen. Angriffe haben diese Eigenschaft, wenn sich dadurch das Geheimnis (bei Public-Key-Kryptographie der private Schlüssel) berechnen lässt.
        
        Zusätzlich ist noch eine weitere Kategorie verwendbar:
        \paragraph{Chosen Chipher Attack}
            Hierbei kann der Angreifer jeglichen Geheimtext entschlüsseln. Zusätzlich liegen ihm eine beliebige Menge an abgefangenen Geheimtexten zur Verfügung. Dabei ist die natürlich die Vertraulichkeit bereits versandter Nachrichten kompromittiert. Jedoch ist hierbei das Ziel des Angriffs, das verwendete Geheimnis zu berechnen.
            
        Nur wenn für kryptographische Verfahren keiner der aufgeführten Stufen an Voraussetzungen ausreicht, um das verwendete Geheimnis zu berechnen sind diese als sicher zu betrachten. In dieser Arbeit wird mit kryptographischen Verfahren gearbeitet, die als sicher betrachtet werden können.

    \subsection{Prinzipien}
        In der Kryptologie gelten verschiedene Prinzipien. Diese sind zwar in der theoretischen Betrachtung nicht notwendig, haben aber in der realen Welt eine große Bedeutung.

            % Rotierende Schlüssel, PERFECT FORWARD SECURITY
        \subsubsection{Prefect Forward Security}
            \ac{PFS} ist ein Prinzip für kryptographische Verfahren, dass durch zukünftige Veröffentlichung des Geheimnisses, die Vertraulichkeit von in der Vergangenheit versandten Nachrichten nicht gefährdet ist. Dies wird garantiert dadurch, dass langlebige Geheimnisse (bzgl. der Speicherung und Nutzung) zusammen mit temporären Geheimnissen zur Verschlüsselung genutzt werden. Somit kann ein Angreifer, der alte Geheimtexte gesammelt hat und im Besitz des langlebigen Geheimnisses ist, die gespeicherten Geheimtexte nicht entschlüssel. Dies kann auch z.B. durch rotierende Geheimnisse, wie Sitzungsschlüssel erreicht werden.  

        \subsubsection{Prinzip von Kerckhoff}
            Das Prinzip von Kerckhoff besagt, dass die Sicherheit eines kryptographischen Verfahrens nicht auf der Geheimhaltung des Algorithmus beruhen darf. Dabei soll die Sicherheit alleinig auf dem verwendeten Geheimnisses und seiner Geheimhaltung beruhen. Natürlich sind Verfahren denkbar, die gegen Kerckhoff's Prinzip verstoßen denkbar, aber auf Grundlage der geschichtlicher Erkenntnisse zu vermeiden. \footcite[19]{Beutelspacher.2015}


\section{Mathematik}
    Mathematische Probleme stellen die Grundlage für moderne Kryptographie. 

    \subsection{Diskreter Logarithmus}
    \label{sec-Diskreter Lograithmus}
        Bei der Bestimmung des Logarithmus wird der Exponent (hier: $x$) gesucht, welcher mit einer bekannten Zahl als Basis $z$, eine weitere bekannte Zahl $y$ ergibt.
        \begin{equation}
            z^{x} = y
            \eqlabel{eq-log}{Normaler Logarithmus}
        \end{equation}

        Der diskrete Logarithmus bezieht hier auf die Berechnung des Logarithmus in ein Gruppe. Diese Gruppe bildet sich aus der Rechnung mit Restklassen (modulo). Dadurch entsteht folgendes Problem, bei der die Variable $x$ gesucht ist und alle anderen Variablen bekannt sind.
        \begin{equation}
            z^{x} \pmod n \equiv y
            \eqlabel{eq-dis-log}{Diskreter Logarithmus}
        \end{equation}
        
        Hierbei ist in der Notation zu beachten, dass sich durch das Rechnen auf mit einer Gruppe, Äquivalenzklassen ($\equiv$) bilden. Diese entsprechen den Restklassen des Rechnen mit Modulo. $n$ ist die Mächtigkeit der Aquivalenzklassen.

        Die Bestimmung von $x$ in \ref{eq-dis-log} wird als Problem des diskreten Logarithmus bezeichnet. Mit der Komplexität wird sich in den Grundlagen der Komplexitätstheorie beschäftigt.

        Dabei ist die Umkehrfunktion, des diskreten Logarithmus $f(x)$ \ref{eq-dis-log}, mathematisch einfach zu berechnen. Diese Umkehrfunktion entspricht der diskreten Exponentialfunktion:
        \begin{equation}
            f^{-1}(x) = z^{x} \pmod n \equiv y
            \eqlabel{eq-dis-exp}{Diskretere Exponentialfunktion}
        \end{equation}
        Hierbei sind $z,x,n$ gegeben und $y$ gesucht.

    \subsection{Faktorisierung}
    \label{sec-Faktorisierung}
        Bei der Faktorisierung wird versucht eine Zahl in Faktoren zu zerlegen. Dabei handelt es sich, im Kontext der Kryptographie, meist um die Faktorisierung des Produkts  zweier großer Primzahlen. Dadurch bildet sich folgende Formel, wobei $p$ und $q$ Primzahlen sind (also Element der Menge der Primzahlen $\mathbb{P}$) und $n$ das resultierende Produkt:
        \begin{equation}
            n = p * q \mid p,q \in \mathbb{P}
            \eqlabel{eq-factor}{Faktorisierung großer Zahlen}
        \end{equation} 
        Da $n$ das Produkt zweier Primzahlen ist, sind seine einzigen Teiler: $n$ selbst, 1 und die seine Primfaktoren $p$ und $q$. Deshalb handelt es sich hierbei auch um eine Primfaktorzerlegung von $n$. 
        
        Dabei ist die Primfaktorzerlegung von $n$ ein rechenaufwändiges Problem, falls $p$ udn $q$ große Zahlen sind. Im Gegensatz dazu ist die Berechnung von bzw. die Validierung mit $n$ sehr einfach, da hierfür nur die Multiplikation von $p$ und $q$ notwendig ist. Somit liegt die gleiche Situation, wie beim Problem des diskreten Logarithmus \ref{sec-Diskreter Lograithmus} vor: Ein rechenaufwändiges Problem, dessen Umkehrfunktion sehr einfach ist \footcite[179]{BSW.2015}. 


    \subsection{Effiziente Berechnung der diskreten Exponentialfunktion}
    \label{sec-Effiziente Berechnung der diskreten Exponentialfunktion}
        In der Kryptographie werden große Zahlen genutzt, um die Sicherheit der verwendeten Algorithmen zu gewährleisten. Hierfür wird als Beispiel angenommen, dass als Basis $z$ eine 256-bit Lange Zahl hoch einem 300-bit langem Exponenten $x$ genommen werden soll. Hierbei ist $n$ 1024-bit lang. 

        Wenn mann nun $z$ in Byte berechnet wäre dies eine 32 Byte lange Zahl.

        $x$ entspricht einer ungefähr 90. stelligen Zahl. 

        \begin{equation}
            z^{10^{90}} \pmod n \equiv y
            \eqlabel{eq-dis-exp-lN}{Diskretere Exponentialfunktion mit großen Zahlen}
        \end{equation}

        Eine numerische Berechnung von $ z^{10^{90}} $ ist aufgrund von begrenzten Ressourcen nicht möglich. 

        Jedoch kann man sich die diskrete Eigenschaft dieser Problems sich zu nutzte machen. Hierfür können Verfahren, wie Square-and-Multiply zusammen mit der Restklassenberechnung genutzt werden. Dadurch lassen sich auch großzahlige Exponenten berechnen. Hierfür soll ein einfaches Beispiel gegeben werden:
        \begin{equation}
            37^{52} \pmod {128} \equiv y
            \eqlabel{eq-dis-exp-bspOne}{Diskretere Exponentialfunktion mit großen Zahlen Beispiel-Eins}
        \end{equation}
        Bei Betrachtung der Äquivalenzgleichung fällt auf, dass $37^{52}$ eine große Zahl ergibt. Jedoch wird diese Zahl noch $ x \pmod 128$ gerechnet. Dadurch liegt das Ergebnis in einem Zahlenraum von:
        \begin{equation}
            y \in \mathbb{N} \mid 0 \le y < 128 
            \eqlabel{eq-dis-exp-numSpace}{Diskretere Exponentialfunktion in Zahlenraum}
        \end{equation}

        Auf Grundlage der Potenzgesetze wird $37^{52}$ nun zerlegt. 
        \begin{equation}
        \begin{aligned}
            52 = 32 + 16 + 4 = 2^{5} + 2^{4} + 2^{2} \\
            37^{52} \pmod {128} \equiv 37^{2^{5}} * 37^{2^{4}} * 37^{2^{2}} \\
            \equiv 37^{2^{5}} \pmod {128} * 37^{2^{4}} \pmod {128} * 37^{2^{2}} \pmod {128}
        \end{aligned}
        \eqlabel{eq-dis-exp-bspTwo}{Diskretere Exponentialfunktion mit großen Zahlen Beispiel-Zwei}
        \end{equation}

        Die einzelnen Bestandteile werden dann iterativ berechnet und durch Multiplikation zusammengefasst (siehe \ref{eq-dis-exp-bspTwo}). Dies wird als Square-and-Multiply-Verfahren bezeichnet.
        \begin{equation}
        \begin{aligned}
            37^{2^{2}} \pmod {128} \equiv (37^{2^{1}} \pmod {128})^{2} \\
            37^{2^{3}} \pmod {128} \equiv (37^{2^{2}} \pmod {128})^{2} \\
            37^{2^{4}} \pmod {128} \equiv (37^{2^{3}} \pmod {128})^{2} \\
            37^{2^{5}} \pmod {128} \equiv (37^{2^{4}} \pmod {128})^{2}
        \end{aligned}
        \eqlabel{eq-dis-exp-bspThree}{Diskretere Exponentialfunktion mit großen Zahlen Beispiel-Drei}
        \end{equation}
        
        \subsubsection{Allgemein}
        Gegeben mit gesucht $y$:
        \begin{equation}
            z^{x} \pmod n \equiv y
        \end{equation}

        Zerlegung von $x$ eine Summe von Zweierpotenzen:
        \begin{equation}
            x = 2^{0} + 2^{1} + 2^{2} + ...
        \end{equation}

        Dabei bilden die binären Logarithmen der einzelnen Zweierpotenzen die Menge $\mathbb{K}$.

        Berechnung der einzelnen Faktoren durch iteratives Square-and-Multiply-Verfahren. Dies wird bis $f(max(\mathbb{K}))$ berechnet. $max(\mathbb{K})$ steht hier für das Element von $\mathbb{K}$, mit dem größten Wert.
        \begin{equation}
            f(i+1) = f(i)^{2} \pmod n \mid f(1) = z^{1} \pmod n 
        \end{equation}

        Zuletzt wird das Produkt, aller Ergebnisse von $f(x)$ für die Elemente der Menge $\mathbb{K}$, gebildet. Dabei gilt:

        \begin{equation}
            \prod_{k \in \mathbb{K}} f(k) \equiv z^{x} \pmod n \equiv y
        \end{equation}

\section{Komplexitätstheorie}
    Die Komplexitätstheorie befasst sich mit der Komplexität von Problemen, welche durch Algorithmen gelöst werden. Dabei wird der Speicherbedarf und der Zeitaufwand eines Algorithmus. Schrankenfunktionen werden gebildet durch die Betrachtung des Speicherbedarf und des Zeitaufwands im Bezug auf die Länge der Eingabeparameter. Da hier eine reine kryptographische Betrachtung der Schrankenfunktionen stattfinden soll, wird hier nur in zwei verallgemeinerte Schrankenfunktionen \footcite[178]{BSW.2015} unterschieden:
    \begin{itemize}
        \item Polynomiale Komplexität
        \item Nichtpolynomiale Komplexität
    \end{itemize}
    Polynomiale Komplexität umfasst hier alle Probleme, die algorithmisch mit polynomialem Aufwand (Zeit/Speicher) gelöst werden können. D.h. bei steigender Eingabelänge $n$ steigt der Aufwand im schlimmsten Fall mit $\mathcal{O}(n^{c} \mid \text{c is constant})$.
    Diese Probleme gehören damit zur Komplexitätsklasse \textbf{P}. Diese umfasst alle Probleme, welche algorithmisch mit maximal polynomialem Aufwand gelöst werden können. Diese Probleme können meistens von modernen Computern gelöst werden.

    Nichtpolynomiale Komplexität hingegen umfasst alle Probleme, die mehr als polynomialen Aufwand im Worst-Case brauchen. Dies können Probleme sein, die algorithmisch nur mit exponentiellen $\mathcal{O}(d^{n} \mid d > 1)$ oder faktoriellen $\mathcal{O}(n!)$ Aufwand \footcite{wiki.komplex} gelöst werden können. 
    Diese Probleme werden der Komplexitätsklasse \textbf{NP} zugewiesen. Dies sind Probleme können nicht von deterministischen Computern in mit polynomialen Aufwand gelöst werden. 
    
    \paragraph{Bezug zur Kryptographie} Dadurch sind sie für die Kryptographie besonders interessant, da man die Sicherheit eines Systems auf ein NP-vollständiges Problem stützen kann. Somit ist die theoretische Sicherheit des Systems nicht brechbar. Jedoch sollte darauf geachtet werden, dass die vorgesehenen Teilnehmer an einem Datenaustausch nicht auch das NP-vollständige problem lösen müssen. Ihr Aufwand soll so gering wie möglich gehalten werden, wobei der Aufwand für einen Angreifer exponentiell oder faktoriell zur Sicherheit der Systems (z.B. die Länge des Schlüssels) ist. 

    Beispiele für solche Probleme sind der diskrete Logarithmus \ref{sec-Diskreter Lograithmus} und die Faktorisierung \ref{sec-Faktorisierung} eines Produkt von Primzahlen\footcite[179]{BSW.2015}. Weitere Beispiele wäre die Berechnung des Isomorphismus zweier Graphen, das Berechnen von Modularen Quadratwurzeln oder die Multiplikation auf elliptischen Kurven.



