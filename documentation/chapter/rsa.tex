\chapter{RSA}
\ac{RSA} ist ein kryptographisches Verfahren, welches zu den Public-Key-Verfahren gehört. Der Verfahren wurde von R. Rivest, A. Shamir und L. Adleman erfunde und trägt deshalb ein Anagram Ihrer Namen.

\section{Sicherheit von RSA}
Die Sicherheit des RSA-Verfahrens basiert auf zwei mathematischen Problemen, welche unter Aufwand endlicher Ressourcen, nicht gelöst werden können. Hierbei wird sich sowohl auf RSA-gestützte Verschlüsselungs- und Signaturverfahren bezogen.
Diese zwei Probleme sind:
\begin{itemize}
    \item Faktorisierung einer bekannten Zahl, welche das Produkt zweier großer Primzahlen ist. Im Kontext von RSA ist diese Zahl mit $n$ repraesentiert.
    \item Bestimmung des diskreten Logarithmus. Bei RSA wäre dies die Bestimmung von 
    \begin{equation}
        d \mid m^{d} \equiv c \pmod n .
    \end{equation}
\end{itemize}

Für die Sicherheit der Public-Key-Verschlüsselung von RSA, spielt Unberechenbarkeit der Faktorisierung die Hauptrolle. Falls mit RSA signiert werden soll, ist zusätzlich die Unberechenbarkeit des diskreten Logarithmus wichtig. Ansonsten könnte der private und geheime Schlüssel abgeleitet werden.



