\chapter{RSA}
\ac{RSA} ist ein kryptographisches Verfahren, welches zu den Public-Key-Verfahren gehört. Der Verfahren wurde von R. Rivest, A. Shamir und L. Adleman entwickelt und trägt deshalb ein Anagram der Erfinder als Namen.

\section{Ablauf}
    \paragraph{Ausgangszenario} Teilnehmer A will über ein öffentliches Netz sicher mit anderen Teilnehmer kommunizieren. 

    \paragraph{Schlüsselgeneration} Damit andere Teilnehmer geheime Nachrichten schicken können muss A sich ein Schlüsselpaar generieren. Dafür wählt er zwei zufällige und große Primzahlen: $p$ und $q$. 

    Das Produkt von $p$ und $q$ bildet $n$, welche den Modulo / den Zahlenraum für weitere mathematische Operationen festlegt: 
        \begin{equation}
            n = p * q
            \eqlabel{eq-rsa-npq}{RSA - Primfaktoren}
        \end{equation}

    Daraufhin berechnet der Teilnehmer die Eulersche $\phi$-Funktion von $p$ und $q$:
        \begin{equation}
            \phi(n) = (p-1) * (q-1)
            \eqlabel{eq-rsa-phi_n}{RSA - Eulersche $\phi$-Funktion}
        \end{equation})
    
    Der öffentliche Schlüssel $e$ ist dann eine zu $\phi(n)$ teilerfremde Zahl. Man kann dies auch vereinfachen und eine Fermat'sche Primzahl für $e$ verwenden:
        \begin{equation}
            2^{2^{n}}+1 \mid n \in \{0,1,2,3,4\}
            \eqlabel{eq-rsa-ferma}{Fermat'sche Primzahl}
        \end{equation}
    
    Der größte gemeinsame Teiler von $e$ und $\phi(n)$ bildet $d$, den privaten Schlüssel. Dabei gilt:
        \begin{equation}
            c * \phi(n) + e * d \equiv 1
            \eqlabel{eq-rsa-calc_d}{RSA - Berechnung des privaten Schlüssels}
        \end{equation}
    
    Danach besitzt der Teilnehmer A ein öffentlichen Schlüssel $e$ und einen privaten Schlüssel $d$. Er kann nun den öffentlichen Schlüssel $e$ zusammen mit $n$ veröffentlichen. $e$ zusammen mit $n$ und $d$ sind ein Schlüsselpaar welches in einen öffentlichen Teil und ein privaten Teil geteilt wird. Dabei herrscht eine eindeutige Zuordnung zwischen den Teilen.
    
    \subsection{Verschlüsselung}
        Wenn Teilnehmer B eine verschlüsselte Nachricht $m$ an Teilnehmer A senden will, braucht er hierfür den öffentlichen Schlüssel von A. Dabei B  verschlüsselt wie folgt, wobei $c$ der resultierende Geheimtext ist.

        \begin{equation}
            m^{e_{A}} \pmod n = c
            \eqlabel{eq-rsa-encrypt}{RSA - Verschlüsselung einer Nachricht}
        \end{equation}

        Wenn Teilnehmer A den Geheimtext $c$ von B erhält, entschlüsselt er diesen mit seinem privaten Schlüssel. Dadurch berechnet er die von Teilnehmer B verschlüsselte Nachricht $m$.

        \begin{equation}
            c^{d_{A}} \pmod n = m
            \eqlabel{eq-rsa-decrypt}{RSA - Entschlüsselung eines Geheimtextes}
        \end{equation}

        Da $d$ privat und eindeutig für den öffentlichen Teil des Schlüsselpaars ist, können nur Teilnehmer, die über $d$ verfügen, Nachrichten entschlüsseln, die mit dem zugehörigen öffentlichen Teil verschlüsselt wurden.
    
    \subsection{Signatur}
        Das RSA-Verfahren ermöglicht auch das Signieren von Nachrichten. Dies wird z.B. im Internet genutzt um Teilnehmer zu authentifizieren und Nachrichtenintegrität zu sichern. RSA wird jedoch meist nur auf kleinere Nachrichten wie Fingerabdrücke angewandt.

        Signatur $s$ einer Nachricht $m$ von Teilnehmer A:
        \begin{equation}
            m^{d_{A}} \pmod n = s
            \eqlabel{eq-rsa-sign}{RSA - Signieren einer Nachricht}
        \end{equation}

        Zum Überprüfen der Echtheit braucht der Teilnehmer B den öffentlichen Schlüssel von A:
        \begin{equation}
            s^{e_{A}} \pmod n = m
            \eqlabel{eq-rsa-check_sign}{RSA - Prüfen einer Signatur}
        \end{equation}

        Eine Signatur kann eindeutig Teilnehmer A zugeordnet werden, da nur ein Teilnehmer im Besitz von $d_{A}$ eine Signatur einer Nachricht erstellen kann, die mit dem öffentlichen Teil des Schlüsselpaars von A überprüft werden kann.
    
    

\section{Sicherheit}
    Die Sicherheit des RSA-Verfahrens basiert auf zwei mathematischen Problemen, welche unter Aufwand endlicher Ressourcen, nicht gelöst werden können. Hierbei wird sich sowohl auf RSA-gestützte Verschlüsselungs- und Signaturverfahren bezogen.
    Diese zwei Probleme sind:
    \begin{itemize}
        \item Faktorisierung einer bekannten Zahl, welche das Produkt zweier großer Primzahlen ist. Im Kontext von RSA ist diese Zahl mit $n$ repräsentiert.
        \item Bestimmung des diskreten Logarithmus. Bei RSA wäre dies die Bestimmung von 
        \begin{equation}
            d \mid m^{d} \equiv c \pmod n .
        \end{equation}
    \end{itemize}

    Für die Sicherheit der Public-Key-Verschlüsselung von RSA, spielt Unberechenbarkeit der Faktorisierung die Hauptrolle. Falls mit RSA signiert werden soll, ist zusätzlich die Unberechenbarkeit des diskreten Logarithmus wichtig. Ansonsten könnte der private Schlüssel bestimmt werden.

    \subsection{Angriffe auf RSA}
    % Momentane Angriffe auf RSA
    % RSA ist momentan als sicher bekannt
    % Erwähnung von Side Channel Angriffen
           


