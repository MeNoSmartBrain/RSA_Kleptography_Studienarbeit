\chapter{Kleptographie}
    \section{Definition}
    % Was ist Kleptographie


    \section{Geschichte}
    % Ursprünglicher Angriff / Konzept

    \section{Angriffskategorie}
        Bisher wurden Angriffe auf kryptographisches Systeme in eine der vier Kategorien \ref{sec-Kryptoanalyse} unterteilt (Known Cipher, Known Plaintext, Chosen Cipher, Chosen Plaintext). Ein kleptographischer Angriff fällt jedoch in keiner dieser Kategorien. Für kleptographische Angriffe müsste eine weitere, fünfte Kategorie geschaffen werden: Known Key Attacks. 
        % Angriffe sind vielleicht Side-Channel Attacks
        % Angriff auf die Implementation

    \section{Aufbau kleptographischer Angriffe}
    

        \subsection{Vorrausetzungen}


        \subsection{SETUP}
        % Beschreibung des SETUPS

    \section{SETUP für RSA}
        \subsection{Voraussetzungen}
            Für ein \ac{SETUP}-Angriff auf eine Implementation von \ac{RSA} hat der Angreifer ein eigenes Schlüsselpaar: $N_{A}$ Modulus des Angreifers, $E_{A}$ Öffentlicher Schlüssel des Angreifers, $D_{A}$ Privater Schlüssel des Angreifers. Das Schlüsselpaar wird wie für \ac{RSA} üblich generiert.
        
        \subsection{Generierung und Verschlüsselung}
            \paragraph{Schritt 1} \label{sec-Schritt-Gen 1} Es wird eine zufällige Primzahl $P$ generiert. $P$ wird dann mit dem öffentlichen Schlüssel des Angreifers verschlüsselt:
            \begin{equation}
                vP = P^{E_{A}} \mod N_{A}
                \eqlabel{eq-SETUP-vP}{Verschlüsselung von P mit dem öffentlichen Schlüssel des Angreifers}
            \end{equation}

            \paragraph{Schritt 2} \label{sec-Schritt-Gen 2} $N'$ wird gebildet indem $vP$ und eine Zufallszahl gleicher Länge $t$ in binärer Form konkateniert werden:
            \begin{equation}
                N' = vP || t
                \eqlabel{eq-SETUP-N'}{Berechnung des temporären Modulus}
            \end{equation}
            $N'$ ist dabei nicht der Modulus des generierten \ac{RSA}-Schlüsselpaars sondern nur eine temporäre Form.

            \paragraph{Schritt 3} \label{sec-Schritt-Gen 3} Berechnung der zweiten Primzahl $Q$: 
            \begin{equation}
                P \cdot Q + R = N'
                \eqlabel{eq-SETUP-Q}{Berechnung der zweiten Primzahl P}
            \end{equation}
            
            \paragraph{Schritt 4} \label{sec-Schritt-Gen 4} Bestimmung des Modulus $N$, wie für \ac{RSA} üblich durch:
            \begin{equation}
                N = P \cdot Q
                \eqlabel{eq-SETUP-N}{Berechnung von N}
            \end{equation}

            \paragraph{Schritt 5} \label{sec-Schritt-Gen 5} Wählen des öffentlichen Schlüssels $E$ und Berechnen des privaten Schlüssels $D$ mittels der modularen multiplikativen Inversen bzgl. $\phi(N)$:
            \begin{equation}
                D = modular\_multiplicative\_inverse(E, \phi(N))
                \eqlabel{eq-SETUP-D}{Berechnung von D}
            \end{equation}

            \paragraph{Schritt 6} \label{sec-Schritt-Gen 6} Mit Schritt 5 wurde ein vollkommen funktionales \ac{RSA}-Schlüsselpaar erstellt. Mittels diesem können nun Informationen verschlüsselt/signiert, Chiffren entschlüsselt und Signaturen verifiziert werden, wie in \ref{sec-RSA-crypt} und \ref{sec-RSA-sign} gezeigt wurde.
        
        \subsection{Angriff}
            \paragraph{Schritt 1} \label{sec-Schritt-Ang 1} Der Angreifer erlangt den öffentlichen Schlüssel des Ziels und besitzt somit $N$ und $E$. Dies ist möglich, da diese Informationen öffentlich sind.

            \paragraph{Schritt 2} \label{sec-Schritt-Ang 2} Der Angreifer teilt $N$ in binärer Form in der Hälfte womit er $vP$ erhält. Die mathematische Begründung hierfür in \ref{sec-SETUP-vP_from_N}.

            \paragraph{Schritt 3} \label{sec-Schritt-Ang 3} $P$ wird durch die Entschlüsslung von $vP$ mittels des privaten Schlüssels des Angreifers berechnet: 
            \begin{equation}
                P = (vP)^{D_{A}} \mod N_{A}
                \eqlabel{eq-SETUP-P}{Berechnung des ersten Primfaktors bei kleinem R}
            \end{equation}
            Damit ist dieser Schritt die inverse Operation zu \ref{sec-Schritt-Gen 1}.
            Zusätzlich muss auch $vP + 1$ entschlüsselt werden.
            Die mathematische Begründung hierfür in \ref{sec-SETUP-Hin-Prim}.
            % Bei der Überprüfung, Prüfen ob P und Q prim
            \begin{equation}
                P' = (vP+1)^{D_{A}} \mod N_{A}
                \eqlabel{eq-SETUP-Palt}{Berechnung des ersten Primfaktors bei großem R}
            \end{equation}
            
            \paragraph{Schritt 4} \label{sec-Schritt-Ang 4} Hiermit ist der Angreifer im Besitz des ersten Primfaktors $P$ oder $P'$. Somit ist die Primfaktorzerlegung von $N$ trivial:
            \begin{equation}
                Q = N / P
                \eqlabel{eq-SETUP-Q}{Primfaktorzerlegung für P}
            \end{equation}
            Die Primfaktorzerlegung muss, gleich wie bei \ref{sec-Schritt-Ang 3}, für den alternativen Primfaktor $P'$ berechnet werden:
            \begin{equation}
                Q' = N / P'
                \eqlabel{eq-SETUP-Q}{Primfaktorzerlegung für P'}
            \end{equation}

            \paragraph{Schritt 5} \label{sec-Schritt-Ang 5} Der Angreifer ist hiermit im Besitz der Primfaktoren $P$, $Q$ und kann den privaten Schlüssel $D$ bestimmen \eqref{eq-SETUP-D}. Gleiches muss für die alternativen Primfaktoren berechnet werden.

            \paragraph{Schritt 6} \label{sec-Schritt-Ang 6} Der Angreifer besitzt den privaten und öffentlichen Schlüssel. Somit können Chiffren entschlüsselt und Signaturen gefälscht werden. Dabei muss der Angreifer, wenn noch nicht geschehen, den privaten Schlüssel $D$ und den alternativen privaten Schlüssel $D'$ einmalig testen, um den richtigen zu bestimmen.

        \subsection{Hintergründe zum RSA-SETUP}
            \subsubsection{Informationsgewinnung von vP aus N} \label{sec-SETUP-vP_from_N}
                

            \subsubsection{Bedigungen für den alternativen Primfaktor} \label{sec-SETUP-Hin-Prim}
                

            \subsection{Verfahren zur Bestimmung des korrekten Primfaktors}
                Die Schritte 3 bis 6 des Angriffs befassen sich mit dem Finden des korrekten Primfaktor aus den zwei resultierenden Möglichkeiten von vP $vP$ und $vP+1$. Daraus werden die Werte und ihre Alternativen für $P$, $N$, $Q$ und $D$ berechnet. 
                Um schlussendlich zu entscheiden, ob die Werte die aus $vP$ oder $vP+1$, kann eine Signatur mit $D$ und $D'$ mit einer Signatur des Angriffsziels verglichen werden. Dadurch kann eine eindeutige Entscheidung getroffen werden. 

                Diese Entscheidung kann jedoch unter Umständen früher berechnet werden. Dieser Fall kann bei folgenden Berechnungsschritten  auftreten:

                

                \subsubsection{Berechnung von P}
                    $P$ wird berechnet indem $vP$ mittels dem privaten Schlüssel $D_{A}$ des Angreifers entschlüsselt wird. Dabei sollte, wie für eine \ac{RSA}-Ver-/Entschlüsselung üblich, eine vollkommen zufällige Zahl resultieren. Jedoch ist es eine Bedingung, dass $P$ prim ist. 
                    Die Wahrscheinlichkeit, dass eine Zufallszahl, mit steigender Anzahl an Stellen, prim ist sehr gering. 
                    Mathematische Erläuterung %ref

                    Falls das entschlüsselte $P$ nicht prim ist, muss es die Alternative $P'$ sein. Gleiches gilt wiederum auch für $P'$.

                \subsubsection{Berechnung von Q} 
                    Bei der Berechnung von $Q$ gilt die gleiche Eigenschaft, wie bei $P$, dass $Q$ prim seien muss.

                    Jedoch kann unter Umständen schon bei der Berechnung von $Q$, durch die Division mit Dividend $N$ und Divisor $P$, die Entscheidung getroffen werden. Dabei wird geprüft, ob $Q$ ganzzahlig ist. 
                    Die Wahrscheinlichkeit für diesen Fall wird hier erläutert: %ref.
                    Die Entscheidung kann hierbei mit einer Laufzeit von $O(1)$ getroffen werden. Dies ist deutlich schneller als die Laufzeit einer Überprüfung auf prim. 
                    Diese Überprüfung wird hier erläutert %ref.

                \subsubsection{Fehler bei der modularen multiplikativen Inverse}
                    Bei der Berechnung von $D$ wird die modulare multiplikative Inverse von $E$ und $\phi(N)$ bestimmt. Dabei kann es zu einem Fehler kommen, da für den Tupel von $E$ und $\phi(N)$ möglicherweise keine modulare multiplikative Inverse existiert.

                    % Mathematische Wahrscheinlichkeit
                
                




        



      


