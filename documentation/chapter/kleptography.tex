\chapter{Kleptographie}
    \section{Definition}
        Klepotgraphie ist ein von Adam Young and Moti Yung geprägter Begriff für das wissenschaftliche Feld der Extraktion von geschützten Informationen. Im Bezug auf kryptographische Systeme sind kleptographische Angriffe die geheime Extraktion vor geheimen Schlüsseln.


    \section{Angriffskategorie}
        Bisher wurden Angriffe auf kryptographisches Systeme in eine der vier Kategorien \ref{sec-Kryptoanalyse} unterteilt (Known Cipher, Known Plaintext, Chosen Cipher, Chosen Plaintext). Ein kleptographischer Angriff fällt jedoch in keiner dieser Kategorien. Für kleptographische Angriffe müsste eine weitere, fünfte Kategorie geschaffen werden: Controlled Key Attacks. 
        Dabei ist nach einem erfolgreichem Angriff der Schlüssel bekannt. Da nach Kerckhoff's Prinzip die Sicherheit eines Systems ausschließlich auf der Geheimhaltung der Schlüssel basiert, ist somit ein kryptographisches System gebrochen. Die Bezeichnung Controlled Key Attack, wurde gewählt, da der Angreifer die Schlüsselerstellung kontrollieren kann, jedoch nicht den schlüssel an sich direkt übertragen kann.

    \section{Aufbau kleptographischer Angriffe}
    
        \subsection{Voraussetzungen} 
        Die Voraussetzungen für den hier beschriebenen kleptographischen Angriff ist die einmalige Manipulation eines kryptographischen Systems zur Erzeugung von \ac{RSA}-Schlüsselpaaren. Dies kann während der Herstellung oder des Betriebs des Systems gelingen. Dabei muss nur die Schlüsselerstellung manipuliert werden. Die Funktionen zum Signieren, Verifizieren, Ver- und Entschlüsseln des Systems sind davon sich betroffen.


        \subsection{Secretly Embedded Trapdoor with Universal Protection}
        \ac{SETUP} ist ein Angriffskonzept auf Implementationen von (kryptographischen) Algorithmen. Dabei wird entweder mittels symmetrischer oder asymmetrischer Verschlüsselung eine Implementation so manipuliert werden, sodass es dem Angreifer möglich ist die Geheimnisse des Systems, ohne das Berechnen algorithmisch aufwendiger mathematischer Probleme, in Erfahrung zu bringen \footcite[1]{markelova.2020}. Dabei wird versucht die Geheimnisse verschlüsselt an den Angreifer zu übergeben. Dies kann wie hier bei \ac{RSA} über die Manipulation von öffentlichen, eigentlich zufälligen Parameter geschehen. Das Manipulieren von Daten, die der Nutzer selbst veröffentlicht, vermeidet das Verwenden von geheimen Kanälen über die Geheimnisse an den Angreifer geleitet werden, welche durch Analyse von z.B. Netzwerkverkehr aufgedeckt werden können. Der hier verwendete Angriff verwendet asymmetrische Kryptographie. Dies unterstützt die Eigenschaft der "Universal Protection" noch mehr indem die Daten asymmetrisch mit einem öffentlichen Schlüssel des Angreifers verschlüsselt werden. Dadurch ist es auch für den Fall, dass der asymmetrische \ac{SETUP}-Angriff aufgedeckt wird, nur für den Angreifer möglich die Backdoor auszunutzen, da nur er in Besitz des zugehörigen privaten Schlüssels ist.

        Kleptographische bzw. \ac{SETUP}-Angriffe sind bisher theoretische Angriffe auf Implementationen von kryptographischen Algorithmen. Die bisherige Forschung bezieht sich auf Computer-Bausteine wie \ac{TPM}. Solche Bausteine werden z.B. von Motherboard-Herstellern entwickelt um kryptographische Operation zu berechnen und einen nicht einlesbaren Speicher für kryptographische Schlüssel zu stellen. Durch \ac{TPM} können Schlüssel auf diesen erstellt, gespeichert und verwendet werden, ohne, dass z.B. das Betriebssystem diese auslesen kann. Dadurch funktionieren \ac{TPM} also Black-Box, welche nur durch API-Aufrufe erreicht werden können. Der Nachteil an diesem System ist, dass der Nutzer eines TPM dem Hersteller blind vertrauen muss. Wenn ein Hersteller einen \ac{TPM} mit einer \ac{SETUP} versieht oder dazu gezwungen wird, sind alle generierten Schlüssel kompromittiert ohne das der Nutzer Einblick in die Black-Box hat. Dabei kann eine kleptographischer Angriff alleinig durch die Benutzung eines schwachen Zufallszahlengenerators erfolgen, wie z.B. Dual\_EC\_DRBG. 

        Kleptographisch Angriffe auf software-seitige Implementationen von kryptographischen Algorithmen haben den Vorteil einer Black-Box nicht. Jedoch ist die Distribution kryptographischer Software dezentraler im Gegensatz zu der Herstellung von \ac{TPM}. Dabei ist die Verwaltung und Generierung der Schlüssel Hardware unabhängig und alleinig durch Software gelöst. Dadurch sind die Schlüssel zwar nicht so gut geschützt, jedoch können  Nutzer selbst den Quellcode der Krypto-Bibliotheken überprüfen. Software-seitige \ac{SETUP}-Angriffe können also leichter erkannt werden. Die Universal Protection bleibt dennoch bestehen. Allerdings hat Software-Kleptographie den Vorteil aus sich des Angreifers, dass er nicht die Herstellung von \ac{TPM} manipulieren / beeinflussen muss, sondern mittels Quellcode-Viren oder Dependency Confusion Systeme infizieren kann.

    \section{Secretly Embedded Trapdoor with Universal Protection für RSA}
        Der folgende Abschnitt zeigt einen kleptographischen Angriff auf eine Implementation des \ac{RSA}-Algorithmus. Dabei wird die in der Literatur gegebene Vorgehensweise in vier Schritte aufgegliedert:
        \begin{enumerate}
            \item Die Voraussetzung für einen erfolgreichen Angriff und die gegebenen Werte.
            \item Der Angriff, wobei die Parameter des Systems manipuliert werden.
            \item Die gängige Generierung des Schlüsselpaars aus den manipulierten Werten.
            \item Die Vorgehensweise eines Angreifers, welcher aus den öffentlichen Parameter des Schlüsselpaars, die Primfaktoren ableitet und das gesamte Schlüsselpaar rekonstruiert.
        \end{enumerate}
        Darauf werden die in den einzelnen Schritten verwendeten Parameter mathematisch definiert. Dabei sind die Definitionen nicht ausschließlich der Literatur entnommen. Eigene Definitionen wurden hinzugefügt um Grenzfälle auszuschließen, welche in der Literatur nicht beachtet werden. Dies soll die Implementation und die Reproduzierbarkeit erleichtern und verbessern. 
        Zuletzt werden Hintergründe zur Implementierung des SETUPS beschrieben. Diese sollen die Vorgehensweise und die Korrektheit des Verfahrens aufzeigen.

        Hierbei ist es wichtig zu sagen, dass die hier aufgezeigte Vorgehensweise nur zu Teilen der später folgenden Implementation entspricht. Dies ist der Fall, da die hier beschriebene Vorgehensweise wenig bis gar nicht optimiert ist.

        \subsection{Voraussetzungen}
            Für ein \ac{SETUP}-Angriff auf eine Implementation von \ac{RSA} hat der Angreifer ein eigenes Schlüsselpaar: $N_{A}$ Modulus des Angreifers, $E_{A}$ Öffentlicher Schlüssel des Angreifers, $D_{A}$ Privater Schlüssel des Angreifers. Das Schlüsselpaar wird wie für \ac{RSA} üblich generiert.
        
        \subsection{Angriff auf die Schlüsselgenerierung}
            \paragraph{Schritt 1} \label{sec-Schritt-Gen 1} Es wird eine zufällige Primzahl $P$ generiert. $P$ wird dann mit dem öffentlichen Schlüssel des Angreifers verschlüsselt:
            \begin{equation}
                vP = P^{E_{A}} \mod N_{A}
                \eqlabel{eq-SETUP-vP}{Verschlüsselung von P mit dem öffentlichen Schlüssel des Angreifers}
            \end{equation}

            \paragraph{Schritt 2} \label{sec-Schritt-Gen 2} $N'$ wird gebildet indem $vP$ und eine Zufallszahl gleicher Länge $t$ in binärer Form konkateniert werden:
            \begin{equation}
                N' = vP || t
                \eqlabel{eq-SETUP-N'}{Berechnung des temporären Modulus}
            \end{equation}
            $N'$ ist dabei nicht der Modulus des generierten \ac{RSA}-Schlüsselpaars sondern nur eine temporäre Form.

            \paragraph{Schritt 3} \label{sec-Schritt-Gen 3} Berechnung der zweiten Primzahl $Q$: 
            \begin{equation}
                P \cdot Q + R = N'
                \eqlabel{eq-SETUP-Q}{Berechnung der zweiten Primzahl P}
            \end{equation}

            Ab diesem Schritt ist die Generierung des \ac{RSA}-Schlüsselpaar identisch zur Generierung in einem unkomprimierten Kryptosystem. Hierzu ist anzumerken, dass die kompromittierten Parametern nicht von unkompromittierten Parametern zu unterscheiden sind. Dies bildet die Grundlage des Prinzips eines \ac{SETUP}. Es ist anzumerken, dass für gegebene Parameter $P, N'$ diese Gleichung keine Lösung hat. Dies liegt unter anderem am Definitionsbereich von $R$ (siehe \ref{Parameterdefinitionen}). Falls dieser Fall eintritt soll nach Literatur mit \ref{sec-Schritt-Gen 1} wieder begonnen werden.
        
        \subsection{Schlüsselgenerierung mit kompromittierten Parametern}
            \paragraph{Schritt 1} \label{sec-Schritt-Gen 4} Bestimmung des Modulus $N$, wie für \ac{RSA} üblich durch:
            \begin{equation}
                N = P \cdot Q
                \eqlabel{eq-SETUP-N}{Berechnung von N}
            \end{equation}

            \paragraph{Schritt 2} \label{sec-Schritt-Gen 5} Wählen des öffentlichen Schlüssels $E$ und Berechnen des privaten Schlüssels $D$ mittels der modularen multiplikativen Inversen bzgl. $\phi(N)$:
            \begin{equation}
                D = modular\_multiplicative\_inverse(E, \phi(N))
                \eqlabel{eq-SETUP-D}{Berechnung von D}
            \end{equation}

            \paragraph{Schritt 3} \label{sec-Schritt-Gen 6} Mit Schritt 5 wurde ein vollkommen funktionales \ac{RSA}-Schlüsselpaar erstellt. Mittels diesem können nun Informationen verschlüsselt/signiert, Chiffren entschlüsselt und Signaturen verifiziert werden, wie in \ref{sec-RSA-crypt} und \ref{sec-RSA-sign} gezeigt wurde.
        
        \subsection{Extraktion der geheimen Parameter}
            Die folgenden Schritte erläutern, wie der Angreifer aus den 
            \paragraph{Schritt 1} \label{sec-Schritt-Ang 1} Der Angreifer erlangt den öffentlichen Schlüssel des Ziels und besitzt somit $N$ und $E$. Dies ist möglich, da diese Informationen öffentlich sind.

            \paragraph{Schritt 2} \label{sec-Schritt-Ang 2} Der Angreifer teilt $N$ in binärer Form in der Hälfte womit er $vP$ erhält. Die mathematische Begründung hierfür in \ref{sec-SETUP-vP_from_N}.

            \paragraph{Schritt 3} \label{sec-Schritt-Ang 3} $P$ wird durch die Entschlüsslung von $vP$ mittels des privaten Schlüssels des Angreifers berechnet: 
            \begin{equation}
                P = vP^{D_{A}} \mod N_{A}
                \eqlabel{eq-SETUP-P}{Berechnung des ersten Primfaktors bei kleinem R}
            \end{equation}
            Damit ist dieser Schritt die inverse Operation zu \ref{sec-Schritt-Gen 1}.
            Zusätzlich muss auch $vP + 1$ entschlüsselt werden.
            Die mathematische Begründung hierfür in \ref{sec-SETUP-vP_from_N}.
            % Bei der Überprüfung, Prüfen ob P und Q prim
            \begin{equation}
                P' = (vP+1)^{D_{A}} \mod N_{A}
                \eqlabel{eq-SETUP-Palt}{Berechnung des ersten Primfaktors bei großem R}
            \end{equation}
            
            \paragraph{Schritt 4} \label{sec-Schritt-Ang 4} Hiermit ist der Angreifer im Besitz des ersten Primfaktors $P$ oder $P'$. Somit ist die Primfaktorzerlegung von $N$ trivial:
            \begin{equation}
                Q = N / P
                \eqlabel{eq-SETUP-Q1}{Primfaktorzerlegung für P}
            \end{equation}
            Die Primfaktorzerlegung muss, gleich wie bei \ref{sec-Schritt-Ang 3}, für den alternativen Primfaktor $P'$ berechnet werden:
            \begin{equation}
                Q' = N / P'
                \eqlabel{eq-SETUP-Q2}{Primfaktorzerlegung für P'}
            \end{equation}
            Durch die Optimierung in \ref{sec-kep-optQ} kann der richtige Primfaktor bestimmt werden.

            \paragraph{Schritt 5} \label{sec-Schritt-Ang 5} Der Angreifer ist hiermit im Besitz der Primfaktoren $P$, $Q$ und kann den privaten Schlüssel $D$ bestimmen \eqref{eq-SETUP-D}. Gleiches muss für die alternativen Primfaktoren berechnet werden.

            \paragraph{Schritt 6} \label{sec-Schritt-Ang 6} Der Angreifer besitzt den privaten und öffentlichen Schlüssel. Somit können Chiffren entschlüsselt und Signaturen gefälscht werden. Dabei muss der Angreifer, wenn noch nicht geschehen, den privaten Schlüssel $D$ und den alternativen privaten Schlüssel $D'$ einmalig testen, um den richtigen zu bestimmen.
        
        \subsection{Parameterdefinitionen} \label{Parameterdefinitionen}
            In diesem Abschnitt werden die einzelnen Parameter und deren Bedingungen definiert. Diese Definitionen könne auch aus den obigen Schritten entnommen werden. Es wird sich hier für eine wiederholte Aufführung entschieden, um mit einer Übersicht der Verständlichkeit zu helfen, die Reproduzierbarkeit zu steigern und Definitionslücken klarzustellen, welche in Literatur zu diesem Angriff gefunden wurden. Wenn die Definitionslücken in der Implementation nicht beachten werden, kann es zu Fehlern bei der Ausführung kommen. Für die Notation beachte \ref{sec-prim}, \ref{sec-bas-concat} und \ref{sec-bas-lenBin}.

            \paragraph{Schlüssel des Angreifer} muss die halbe Länge des Schlüsselgenerators haben.
                \begin{equation}
                    N_{A} \mid \overline{N_{A}} = n_{A} = \frac{n}{2}
                \end{equation}

            \paragraph{P} ist einer der beiden Primfaktoren des \ac{RSA}-Schlüsselgenerators. 
                \begin{equation}
                    P \in \mathbb{P}, \overline{P} = \frac{n}{2}
                \end{equation}
            
            \paragraph{vP} ist die Verschlüsselung von $P$ mittels dem öffentlichen Schlüssel des Angreifers. Da der Schlüssel des Angreifers $\overline{N_{A}} = n_{A} = \frac{n}{2}$ hat, gilt $\overline{vP} = \overline{P}$. Dabei kann $vP$ als Zufallszahl betrachtet werden (siehe \ref{sec-RSA-assertion}). 
                \begin{equation}
                    P^{E_{A}} \mod N_{A}
                \end{equation}                
            
            \paragraph{N'} $N'$ ist die Konkatenation von $vP$ einer Malware $\frac{n}{2}$ Zufallszahl $t$.
            \begin{equation}
                N' = vP || t
            \end{equation}

            \paragraph{Q} ist der zweite Primfaktor. Jedoch wird er nicht wie üblich generiert, sondern aus den Parametern $P$ und $N'$ berechnet.
            
            \begin{equation}
                Q : N' = P \cdot Q + R \mid Q \in \mathbb{P}, \overline{R} = \frac{n}{2}
            \end{equation}

            \paragraph{N} $N$ ist das Produkt der Primzahl $P$ und $Q$. Dadurch sind die einzigen Teiler von $N$, $P$ oder $Q$. Dabei gilt $\overline{N} = n$.

            \paragraph{E} ist der öffentliche Schlüssel (in Kombination mit $N$) und muss eine Primzahl sein, also $P \in \mathbb{P}$. Dabei kann als eine Fermat'sche Primzahl (siehe \ref{eq-rsa-ferma}) gewählt werden.

            \paragraph{D} ist der private Schlüssel. $mmi()$ ist dabei die modulare multiplikative Inverse.
            \begin{equation}
                D = mmi(E, \Phi(N)) \mid \Phi(N) = (P-1)\cdot(Q-1)
            \end{equation}

            \paragraph{M, C, S}, also jegliche Eingabe in eine \ac{RSA}-Operation muss kleiner $N$ sein.

        \subsection{Hintergründe zum RSA-SETUP}
            \subsubsection{Informationsgewinnung von vP aus N} \label{sec-SETUP-vP_from_N}
            Längendefinitionen:
            \begin{itemize}
                \item $N, N'$ hat $n$ Bit
                \item $vp, R$ hat $\frac{n}{2}$ Bit
            \end{itemize}
            $vP$ kann aus $N$ bestimmt werden, durch die Beziehung von $N$ und $N'$. Dabei gilt $N + R = N'$, wobei $R$ dem Angreifer nicht bekannt ist. Jedoch ist definiert, dass $R$ nur halb so viele Bits wie $N$ hat. Dadurch findet die Addition von $N + R$ nur auf den niederwertigeren Bits von $N$ statt. Somit sind die höherwertigen Bits von $N$ (Bits > $\frac{n}{2}$ nicht von der Addition mit $R$ beeinflusst und somit gleich zu den höherwertigen Bits von $N'$. Dies ist nur dann nicht der Fall, wenn es bei der Addition zu einem Überlauf für. In diesem Fall sind alle Bit > $\frac{n}{2}$ von $N$, die höherwertigen Bits von $N' + 1$.
            Zudem ist definiert, dass die höherwertigen von $N'$ gleich $vP$ sind (siehe \ref{eq-SETUP-N'}).
            Da der Angreifer nur $N$ aber nicht auch $R$ kennt, muss die Möglichkeit eines Überlauf durch $R$ berücksichtigt werden. 
            Aufgrund dessen kann der Angreifer $vP$ aus den höherwertigen Bits von $N$ lesen. Muss aber auch $vP + 1$ als Option berücksichtigen, da er nicht weiß, ob ein Überlauf stattgefunden hat.               
 
